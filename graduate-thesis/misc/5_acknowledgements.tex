\begin{acknowledgements}
    时间飞逝,转眼临近研究生毕业。当这篇文章正式落笔,就意味着我在北京理工大学的时光走到了尾声。回顾这段时间的工作学习,在课堂上学到的理论知识被强化、加深、理解,从而提高了我的专业知识素养,成为我人生的重要财富。在这里,我要向所有帮助过我的老师同学表示衷心的感谢。

    首先,非常真挚地感谢我导师马锐老师。三年时间里,无论是学习还是生活,您都事无巨细,悉心教导。第一个组会PPT、第一次写项目文档、第一次项目汇报、第一次学术交流,诸此种种皆历历在目,每一次都是在您的指导下一遍遍磨合、完善,直至顺利完成。不知不觉间,您认真、严谨的精神已经深深感染了我,让我在学习、工作中能够以更高的标准要求自己。您不仅是我们的良师,更是益友。在日常谈心中,经常以自己的生活经验给与我们最诚恳的建议,也会在我为工作选择而迷茫时,帮我捋清人生方向。感谢您的倾囊相授,让我在三年时间里快速成长,遇见全新的自己。

    感谢420实验室的全体同学对我的鼓励陪伴,是大家让我收获到了友谊。我们在外出求学的生涯里相互扶持,用真挚的心换来坚定的情。虽然我们来自五湖四海,但我们在北京理工大学相遇,北京理工大学也见证了我们一起走过的三年。希望各位即将毕业的同学能够在新的战场上继续发光发热,前途似锦;还继续在校学习的师弟、师妹们能够学业有成,硕果累累。

    感谢北京理工大学对我的培养教育,感谢它为我创造了一个环境优美、治学严谨、学风浓厚的校园氛围。在我迷茫时,可以去图书馆寻找知识解答疑惑;当我伤心时,可以去中心公园平复心情得到治愈;当我开心时,可以去食堂选一份美食来庆祝。我在北京理工大学度过了愉快、幸福的三年,在这里,也希望北理可以不断进步,未来我可以重回校园,为母校的发展感到无比自豪。
    
    感谢父母对我的支持和照顾,感谢他们多年来无微不至的照顾和坚定不移的支持。不管我在哪里,他们永远是我最坚强的后盾。当我遇到困难时,给我最温柔的建议和最坚定的鼓励。无论是在物质上,还是精神上,他们的支持与付出,鼓励着我坚定求学的决心,无畏地继续奔向美好地未来。   
        
    最后,谨向论文评审和答辩中的每一位老师致以深深的敬意,衷心感谢老师们的辛勤付出。再次向各位表示真挚的谢意,感谢在我前进的路上,有你们的参与!

\end{acknowledgements}
