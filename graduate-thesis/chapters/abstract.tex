\begin{abstract}
  代码克隆检测是软件工程领域的重要任务,如何对源代码进行表征学习决定了对源代码表征抽取的程度,进而影响下游任务所能检测的精度。作为代码克隆检测任务的核心技术和研究热点,现有的代码表征学习研究存在诸多不足,例如对代码结构信息和语义信息利用不充分,特征表达不够完善;表征模型对数据集、模型结构和优化算法等多方面因素的要求高等,这些不足导致代码克隆检测效率较低。本文提出了一种多维源代码表征学习方法,旨在通过构建三个不同维度的代码表征模型,将源代码的语义信息表示为稠密低维实值向量,以在低维空间中高效计算实体和关系的语义联系,并通过特征融合得到多维特征,实现对代码信息的充分利用,以更加全面准确与智能化的方式提高代码克隆测试效率。
\end{abstract}

\begin{abstractEn}
  Code cloning detection is an important task in the field of software engineering. How to learn representation of source code determines the degree of source code representation extraction, which in turn affects the accuracy that downstream tasks can detect. As the core technology and research hotspot of code cloning detection task, existing research on code representation learning has many shortcomings, such as insufficient utilization of code structure and semantic information, and incomplete feature expression; The representation model has high requirements for various factors such as dataset, model structure, and optimization algorithms, which leads to low efficiency in code cloning detection. This project proposes a multi-dimensional source code representation learning method, aiming to construct three different dimensional code representation models to represent the semantic information of source code as dense low dimensional real value vectors, efficiently calculate the semantic connections of entities and relationships in low dimensional space, and obtain multi-dimensional features through feature fusion to fully utilize code information and improve the efficiency of code cloning testing in a more comprehensive, accurate, and intelligent way.
\end{abstractEn}
