%%==================================================
%% chapter02.tex for BIT Master Thesis
%% modified by yang yating
%% version: 0.1
%% last update: Dec 25th, 2016
%%==================================================
\chapter{特征融合及RLCCD框架验证}
本章主要对面向代码克隆检测的多维源代码表征方法整体框架RLCCD进行介绍,同时进行实验评估及验证。具体地,RLCCD是由上述基于预训练辅助模型的Token表征学习、基于子树划分的抽象语法树表征学习、基于图过滤的程序依赖图表征学习方法三种维度融合形成并进行实现的表征方法,最后通过与SourcererCC、ASTNN、SCDetector进行对比实验以验证该框架的有效性。
\section{特征融合}
特征融合
\section{RLCCD框架验证}
本文的实验设计主要围绕以下4 个方面的研究问题:

• RQ1:本文提出的预训练辅助模型策略是否优于基线方法?(见3.3 节)

• RQ2:本文提出的子树划分策略是否优于基线方法?(见4.3 节)

• RQ3:本文提出的图过滤策略能否优于基线方法?(见5.3 节)

• RQ4:与现有代码克隆检测工具相比,RLCCD 表现如何?

\subsection{实验设置}
(1)系统环境

本章实验均在Ubuntu 16.04 LTS(64位)系统下进行。


(2)实验数据集

本章实验部分采用的数据集为代码克隆检测领域常见的基准集POJ104,该数据集是一个基于C语言构建的大型数据集,包含了104个编程问题以及学生提交的对应问题的不同C语言源代码,在该数据集中,针对同一问题的不同源解法的代码被视为一个克隆对。

首先通过筛选,得到51485个源代码样本,然后随机生成50000个代码克隆对,其中包含5200个真克隆对,44800个假克隆对。依据随机种子将数据集按照3:1:1划分为训练集、测试机、验证集,其中的正负样本数如下表。


\subsection{对比工具}
在对比整个方法的效果时,本文选取开源的SourcererCC、ASTNN、SCDetector方法进行比较。

SourcererCC:SourcererCC是一种相对较新的基于token的克隆检测工具。该工具通过词袋模型,把收集的数据全部编码成词频信息,后将代码行转换成一个由词频构成的向量,通过向量的比较获取相似度。

ASTNN:一种基于神经网络的源代码表示方法。它将整个抽象语法树AST分解成一系列小型语句子树,并通过捕获语句的词法和语法信息将语句子树分别编码为向量,最后采用了RNN模型生成代码片段的向量表示。ASTNN方法完整保留了抽象语法树的结构信息,能够检测到所有类型的代码克隆。

SCDetector:是基于令牌和基于图的方法的结合。给定一个方法源代码,我们首先生成CFG,然后应用中心性分析将图转换为某些语义标记(即具有图细节的标记)。最后,这些语义标记被馈送到Siamese网络中,以训练模型并使用它来检测代码克隆对。

\subsection{RLCCD性能评估实验结果}

对比实验:体现框架的有效性


\begin{table}
  \centering
  \caption{RLCCD实验结果} %\label{tab:category}
  \begin{tabular*}{0.9\textwidth}{@{\extracolsep{\fill}}cccc}
  \toprule
    对比			&P		&R		&F1 \\
  \midrule
    SourcererCC			&0.xx	&0.xx		&0.xx \\
    ASTNN			&0.xx		&0.xx		&0.xx \\
    SCDetector			&0.xx	&0.xx		&0.xx \\
    RLCDD			&0.xx		&0.xx		&0.xx \\
  \bottomrule
  \end{tabular*}
\end{table}

\section{本章小结}
本章节主要对RLCCD框架进行了实验验证,以验证RLCCD的XXX能力。



