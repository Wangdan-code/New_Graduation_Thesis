\documentclass[type=master,twoside]{bithesis}

% 此处仅列出常用的配置。全部配置用法请见「bithesis.pdf」手册。
\BITSetup{
  cover = {
    %% 使用以下参数来自定义封面日期
    date = 2024年5月,
    autoWidthPadding = 0.25em,
  },
  info = {
    % 想要删除某项封面信息,直接删除该项即可。
    % 想要让某项封面信息留空(但是保留下划线),请传入空白符组成的字符串,如"{~}"。
    % 如需要换行,则用 “\\” 符号分割。
    classification = TQ028.1,
    UDC = 540,
    title = 面向代码克隆检测的多维源代码表征学习方法研究,
    % 如需覆盖竖排标题,请配置以下选项。
    % 下面的例子展示了如何在竖排标题中使用垂直或者旋转的英文。
    % verticalTitle = {形状记忆聚氨酯{L } {T } {X }的合成 \rotatebox[origin=c]{-90}{Feng Kaiyu} 及其在织物中的应用},
    titleEn = Research on Multidimensional Source Code Representation Learning Method for Code Clone Detection,
    author = 王丹,
    % 如果想要手动控制盲审模式下的隐藏信息,可以使用宏 \SecretInfo{}。使用方式有两种,如:
    % major = \SecretInfo{材料科学与工程} 可以得到 ******* (用等量的替换符号替代)
    % major = \SecretInfo{材料科学与工程}[ABCDEF] 可以得到 ABCDEF (用你自定义的内容替代)
    major = 计算机技术,
    school = 计算机学院,
    degree = 工学硕士,
    chairman = **教授,
    defenseDate = 2024年5月1日,
    supervisor = 马锐教授,
    authorEn = Dan Wang,
    schoolEn = Beijing Institute of Technology,
    supervisorEn = Associate Prof. Rui Ma,
    chairmanEn = Prof. **,
    degreeEn = Master of Science,
    majorEn = Computer Technology, 
    defenseDateEn = {May, 12th, 2024},
    keywords = {代码克隆检测;深度学习;代码表征学习},
    keywordsEn = Code cloning detections; Code representation learning; Deep learning
    % 必要时置于封面右上角,并按照国家规定进行标记。
    % classifiedLevel = 密级\BigStar 保密期限,
  },
  % 在目录页中不显示摘要和主要符号对照表的标题。
  TOC = {
    abstract = false,
    abstractEn = false,
    symbols = false,
  },
  style = {
    pageVerticalAlign = scattered,
    % 开启 Windows 平台下的中易宋体伪粗体。
    % windowsSimSunFakeBold = true,
  },
  publications = {
    % 以下两个选项将影响「攻读学位期间发表论文与研究成果清单」中名称列表的省略阈值。
    % 一般来说,如果你在全部文献中最低排在第四位,建议你将两个值都设置为 4。
    % 更详细的说明请见手册。
    maxbibnames = 3,
    minbibnames = 1,
  },
  % 采用章节标题级别的附录格式
  appendices / chapterLevel = true,
  const = {
    % 由于现存的 Word 模板、旧有 LaTeX 模板与《北京理工大学研究生学位论文撰写规范》的规定不一致,
    % 论文封面的某些字段内容需要用户根据自己的情况进行手动调整。
    % 目前给出的默认值是按照 2018 年发布的《北京理工大学研究生学位论文撰写规范》中的要求进行设置的。
    % 比如注释掉的这一行:将会修改封面中的「申请学位级别」为「申请学位」。
    % info / degree = {申\quad{}请\quad{}学\quad{}位},
    % info / major = {学\quad{}科\hspace{5pt}/\hspace{5pt}类\quad{}别}
  }
}

% 大部分关于参考文献样式的修改,都可以通过此处的选项进行配置。
% 详情请搜索「biblatex-gb7714-2015 文档」进行阅读。
\usepackage[
  defernumbers=true,
  backend=biber,
  style=gb7714-2015,
  gbalign=gb7714-2015,
  gbnamefmt=lowercase,
  gbpub=false,
  gbannote=true,
  gbpunctin=false,
  doi=false,
  url=false,
  eprint=false,
  isbn=false,
]{biblatex}

% 添加参考文献
\addbibresource{reference/main.bib}
% 攻读学位期间发表论文与研究成果清单,详细使用方法见 `chapters/pub.tex`。
\addbibresource{reference/pub.bib}


\usepackage{graphicx}

\usepackage{url}
\usepackage{float}
\usepackage{amsmath}
\nocite{*}
\usepackage{multirow}

\begin{document}

% 封面绘制
\MakeCover

% 打印书脊
\MakePaperBack

% 中文信息与英文信息
\MakeTitle

% 论文原创性声明和使用授权
\MakeOriginality

%%%%%%%%%%%%%%%%%%%%%%%%%%%%%%
%% 前置部分
%%%%%%%%%%%%%%%%%%%%%%%%%%%%%%
\frontmatter

% 摘要
\begin{abstract}
  代码克隆检测是软件工程领域的重要任务,如何对源代码进行表征学习决定了对源代码表征抽取的程度,进而影响下游任务所能检测的精度。作为代码克隆检测任务的核心技术和研究热点,现有的代码表征学习研究存在诸多不足,例如对代码结构信息和语义信息利用不充分,特征表达不够完善;表征模型对数据集、模型结构和优化算法等多方面因素的要求高等,这些不足导致代码克隆检测效率较低。本文提出了一种多维源代码表征学习方法,旨在通过构建三个不同维度的代码表征模型,将源代码的语义信息表示为稠密低维实值向量,以在低维空间中高效计算实体和关系的语义联系,并通过特征融合得到多维特征,实现对代码信息的充分利用,以更加全面准确与智能化的方式提高代码克隆测试效率。
\end{abstract}

\begin{abstractEn}
  Code cloning detection is an important task in the field of software engineering. How to learn representation of source code determines the degree of source code representation extraction, which in turn affects the accuracy that downstream tasks can detect. As the core technology and research hotspot of code cloning detection task, existing research on code representation learning has many shortcomings, such as insufficient utilization of code structure and semantic information, and incomplete feature expression; The representation model has high requirements for various factors such as dataset, model structure, and optimization algorithms, which leads to low efficiency in code cloning detection. This project proposes a multi-dimensional source code representation learning method, aiming to construct three different dimensional code representation models to represent the semantic information of source code as dense low dimensional real value vectors, efficiently calculate the semantic connections of entities and relationships in low dimensional space, and obtain multi-dimensional features through feature fusion to fully utilize code information and improve the efficiency of code cloning testing in a more comprehensive, accurate, and intelligent way.
\end{abstractEn}


% 制作目录
\MakeTOC

% 插图目录
\listoffigures
% 表格目录
\listoftables

% 主要符号对照表
% \input{./misc/0_symbols.tex}

\mainmatter

% 请根据论文内容,按照顺序添加章节。
\chapter{绪论}
\label{chap:intro}

\section{研究背景与意义}
\label{sec:background}
代码克隆,也叫代码复用,是指在软件系统中存在两个或两个以上的相似代码片段\cite{乐乔艺2021代码克隆检测研究进展综述},是软件开发中的常见现象。随着互联网时代的发展,网络上各种开源项目越来越多样化,获取也更加便利。许多企业通过软件资源库、外部开源软件、软件产品线及开发框架等方式建立了多种多样的软件复用开发方法,同时开发人员自身也会通过多种方式大量复用已有的软件资源。在这些软件复用方法和资源的支持下,软件系统和软件产品大量引入了开源软件、网络资源、商业软件等第三方代码成分。这些第三方代码在多个软件系统中复制、传播和演化,给软件系统带来了软件质量的不确定性和风险,甚至导致漏洞的传播。

近年来,第三方代码中包含的漏洞数量呈现出快速增长的趋势。根据美国新思科技公司(Synopsys, Inc.)发布的《2023年开源安全和风险分析报告》\cite{Synopsys_2023}显示,在2022年审计的1703个代码库中,98\%的项目都包含开源代码,84\%的代码库包含至少一个已知开源漏洞,比2022年版的报告中增加了近4\%,有48\%代码库中包含高风险漏洞。
图\ref{fig:Proportion}统计了2018年至2022年Synopsys审计代码库中开源代码及漏洞占比,从图中可以看出开源代码及漏洞数量整体呈上升趋势。
\begin{figure}[H]
    \centering
    \includegraphics[width=0.75\textwidth]{figures/Proportion}
    \caption{2018-2022年Synopsys审计代码库中的开源代码及漏洞占比示意图}\label{fig:Proportion}
\end{figure}

同时,Synopsys统计了包含易受攻击组件的代码库占比,其中使用JQuery 和 Lodash 两个最流行的开源组件的代码库占比达到了47\%和 31\%,其余组件占比如图\ref{fig:assembly}所示。一旦易受攻击组件出现安全问题,通常会导致软件遭受供应链攻击。据Gartner\cite{Gartner_2022}预测,到2025年,全球45\%的组织将遭受软件供应链攻击,比2021年增加三倍。因此,准确地检测代码克隆对于软件开发和维护是至关重要的。
\begin{figure}[H]
    \centering
    \includegraphics[width=0.75\textwidth]{figures/assembly}
    \caption{2022年Synopsys审计代码库中包含易受攻击组件的百分比示意图}
    \label{fig:assembly}
\end{figure}

早期进行代码克隆检测通常采用人工检查并标注的方法,通过收集整理大量的代码逐行检查语法、语义结构,由人工复查筛选出正确的克隆代码并对其进行标注,由此形成了早期的代码克隆数据样本,例如,2015年Svajlenko\cite{7332459}等人提出了著名评估基准集BigcloneBench,该数据集是由克隆领域三个专家评委花费216小时通过人工验证的方法从IJaDataset\cite{IJaDataset2.0}中挖掘而来,总数据量达到800万,其背后的人工花费巨大。但利用人工的方法检测代码克隆效率低,成本高,并且无法保证准确率\cite{7965429},因此,有研究人员提出代码克隆检测技术,目的在于自动化定位软件系统中的代码克隆,并能够节约成本,减少出错风险\cite{Yang2015ClassificationMF}。

早期代码克隆检测技术通常将代码视为自然语言文本进行处理,通过文本相似性判断代码相似程度;随着编译技术的发展,研究者们将编译原理中的词法分析技术运用到代码克隆检测领域;近年来,基于多维源代码表征学习的代码克隆检测技术已经引起了学者们广泛的兴趣,有研究人员从代码克隆检测与代码表征学习技术相结合这一方面进行了探索,试图从关键技术点入手,找到合适的结合点,以提高定代码克隆检测技术的效率和智能化程度。


\section{研究现状与趋势}
\label{sec:status}

\subsection{代码克隆检测技术}
\label{subsec:Code clone detection}

代码克隆检测技术,旨在自动化定位软件系统中的代码克隆,节省成本,减少出错风险,有助于更好地保证软件质量。目前已有的代码克隆方法大多需要对代码片段进行信息抽取,转换为中间表征,然后根据表征方式的不同计算不同代码片段之间的相似度,完成克隆检测任务。其具体流程如图\ref{fig:figure1}所示。
\begin{figure}[H]
    \centering
    \includegraphics[width=0.95\textwidth]{figures/figure1}
    \caption{代码克隆检测流程}\label{fig:figure1}
\end{figure}

从图\ref{fig:figure1}可以看出一个完整的代码克隆检测过程通常包括预处理与转换、代码表征、匹配检测、后处理几个阶段。具体而言,一般的代码克隆检测从代码预处理与转换开始,首先删除与检测无关的空白行、注释、缩进等元素,并根据检测粒度将源代码划分为单独的片段,比如类、函数等;然后在代码表征步骤,将比较单元转换为相应的中间表示,常见的中间表示有:词法单元(Token)、抽象语法树(abstract syntax tree,AST)、程序依赖图(Program dependency graph,PDG)等;在匹配检测阶段,将根据得到不同的中间表示采用相应的匹配算法进行相似度计算,例如抽象语法树的比较通常采用子树匹配算法,程序依赖图的比较则采用子图同构算法。此阶段将代码片段两两对比,以查找相似代码源片段,得到代码克隆对。最后在后处理阶段,通常会通过人工检测或者算法过滤掉错误的代码克隆,并以适当的方式呈现给开发人员提供帮助。

在这些步骤中,代码表征方式决定了匹配检测方法的预处理方式、模型设计、部署方式、运行效率,并影响最终结果\cite{陈秋远2019代码克隆检测研究进展}。比如将源代码表征为文本,其预处理过程主要为去除噪声,如空格、注释等,其比较算法可以利用文本相似的一系列方法,能够检测到语法相似的克隆代码;而如果表征为抽象语法树,则其预处理过程需要解释器的参与,相似比较算法更多地考虑了结构相似等,能够检测到语法层面相似的代码克隆。因此, 源代码表征方式是代码克隆检测的关键步骤。


\subsection{代码表征学习}
\label{subsec:Code representation}
表征学习是指学习数据的表示,使其在构建分类器或其他预测因子时更容易提取有用信息\cite{Bengio2013Representation}。代码表征学习是对源代码的语义和语法信息进行表征,得到源代码的特征向量,并将其应用在不同的下游任务上。在代码克隆检测中,代码表征学习可以用来提取代码片段更高层次的抽象特征表示,这些特征表示能够捕捉代码的语法、语义以及结构信息。通过学习到的代码表征,可以更准确地比较和识别不同代码片段之间的相似性,从而实现克隆代码的检测和管理,提高检测的准确性和鲁棒性。因此,代码表征学习为代码克隆检测提供了重要的技术支持。

代码表征学习工作最早可追溯到100年前,,基于传统机器学习的数据特征学习被广泛提出,主成分分析PCA(Principal Component Analysis)\cite{WOS:000202849800065}、线性判别分析LDA(Linear Discriminant Analysis)\cite{2012THE}都是经典的表征学习方法。随着神经网络的不断发展,基于深度学习的代码表征学习工作研究能够更有效地提取数据的特征,用于后续的分类或预测。2013年,Y Bengio等人\cite{Bengio2013Representation}发表了关于表征学习的经典综述。2016年,Bengio和 IGoodfellow等人\cite{goodfellow2016deep}合著的《Deep Leanring》一书中也为表征学习专著一章。近些年来,代码表征学习方法被用于代码克隆检测、代码推荐、代码剽窃等多个代码分析任务中,取得了一定的成就。根据源代码的抽象层次不同,现阶段代码表征学习工作可以分为基于Token的代码表征、基于树的代码表征、基于图的代码表征、基于语法和语义混合的代码表征四类。

(1)基于Token的代码表征

基于Token的代码表征通常利用词法分析器将代码中的词汇单元(Token)划分出来。这些词汇单元通常包含关键字、数字、标识符等。将代码表示为词汇单元序列之后,利用深度学习技术对其进行建模,学习代码序列中所包含的有效信息,如功能语义信息、语法结构信息等,最后生成具有丰富代码信息的表征向量,应用于后续的代码克隆检测任务中。

著名的CCFinder\cite{1019480}、CP-Miner\cite{1610609}等克隆检测工具都是基于Token级的,可以很好地检测完全相同的代码对以及参数化后的代码对克隆问题。其中,CCFinder将源代码中的每一行单独转换为Token序列,根据转化规则对Token进行修改,将类型名、变量名、常量的标识符替换为指定的特殊Token,最后利用后缀树来查找相同的子序列并通过设置阈值来过滤克隆对 。CP Miner增加了Bug检测,该工具的检测速度、检测精度相较于CCFinder有了很大的提高。

Jiang等人\cite{10.1145/1287624.1287634}首先使用神经网络在Token级别进行代码克隆检测,提出CCLearner方法。该方法使用BigCloneBench\cite{7332459}作为训练样本,抽取了其中方法级别的Token序列,将保留字、类型标识符、方法标识符和变量标识符等8种符号表示为8种标记,随后将各种标记类型以及出现的频数作为代码的序列表示,并用于代码克隆检测。

Mikolov等人\cite{pennington-etal-2014-glove}利用Word2vec、GloVe、BERT进行Token的预训练,通过无标注样本训练深度网络结构,使用标注样本进行模型参数微调,从而提升模型性能。其中BERT\cite{devlin-etal-2019-bert}是双向Transformer的编码器,通过遮蔽语言模型和下一句预测2种预训练目标来调整模型参数。

Sajnani等人\cite{7886988}提出了一种基于词袋模型的方法SourcererCC,使用代码段Token的组成来度量两段代码块中词法粒度上的重复度,从而检测两段代码的相似性。这种方法相对于纯文本的代码表示形式实现了更高层次的代码分析。

上述方法均在Token上进行代码的表征学习,力图充分提取代码中的属性信息。

(2)基于树的代码表征

抽象语法树AST是源代码的抽象语法结构的树状表示,可以有效地表示程序的语法及其结构,利用深度神经网络对抽象语法树进行建模得到其向量表示,根据该特征向量完成代码克隆检测任务,实现基于树的源代码表征。

White等人\cite{White2016DeepLC}提出了一种基于循环神经网络的代码表征方法,该方法将代码分为词汇以及句法两个层次。对于词汇级别的信息,该方法在代码的词汇单元序列上使用RNN神经网络进行建模。而对于代码的句法级别的信息,首先将代码转换为其对应的抽象语法树结构,之后将抽象语法树转换为其对应的满二叉树,最后将满二叉树转换为橄榄树,并在其上使用另一个RNN神经网络进行建模。该方法将这两个特征相结合作为整个程序的特征向量,根据该向量进行代码克隆检测任务。

Mou等人\cite{WOS:000485474201046}提出了一种基于树的卷积神经网络模型TBCNN(Tree-based convolutional neural network)。该模型采用了“连续二叉树”的概念,直接在代码所对应的抽象语法树上进行卷积操作。在卷积操作之后获得了不同数目的AST结构特征向量,由于数目不同不能直接作为神经网络的输入,因此该方法还采用“动态池化”技术,最终将数目不同的特征向量转换为了一个向量。TBCNN是一个通用的代码表征生成模型,所生成的向量能够包含代码片段中特有的代码模式,因而可以应用于不同的代码分析任务中。

Wei等人\cite{10.5555/3172077.3172312}提出了一种基于哈希特征的克隆检测方法CDLH(Clone Detection with Learning to Hash)。该方法使用Word2Vec模型学习标记嵌入以捕获词汇信息,然后训练基于抽象语法树的LSTM模型将这些嵌入组合成一个二进制向量来表示代码片段,最后工具通过计算哈希码的汉明距离来检测代码克隆。

Zhang等人\cite{8812062}提出了一种基于抽象语法树的神经网络代码表征方法ASTNN(A novel AST-based Neural Network )。该方法将完整的抽象语法树分割为多个语句子树。针对每个语句树,该方法设计了语句编码器用于将语句树转换为对应的语句表征向量,通过使用双向GRU神经网络对语句向量进行建模,对双向GRU层输出的隐含状态向量进行最大池化操作,以获得最显著的代码特征。该方法所生成的代码表征向量被应用于代码克隆检测任务中,在POJ104\cite{WOS:000485474201046}和BigcloneBench\cite{7332459}数据集上取得了当时最好的检测结果。

Yu等人\cite{8813290}提出了一种基于树卷积的代码克隆检测方法TBCCD(Tree-Based Convolution for Clone Detection)。该方法提出了一种三角形卷积核对父节点和子节点卷积,通过自适应的参数编码树中的节点;同时考虑到抽象语法树的词法信息,通过位置相关的编码方式编码Token值,最后基于CNN进行克隆检测。

上述方法均在抽象语法树上进行代码的表征学习,力图充分提取代码中的结构信息。

(3)基于图的代码表征

程序依赖图PDG是程序的一种图形表示,所含结构信息最多,能够表示程序的控制依赖,数据依赖以及地址依赖等关系,是一种带有标记的有向多重图。通过将程序表示为图的形式使得模型能够更好地理解代码中不同部分之间的依赖关系。

Allamanis等人\cite{Allamanis2017LearningTR}考虑到代码中的长依赖问题,如在代码中变量的定义位置与使用位置之间的距离问题,提出了基于图的代码表征方法,并介绍如何使用GGNN(Gated Graph Neural Networks)训练。首先将代码转换为对应的抽象语法树,之后通过不同的连接规则连接抽象语法树各个节点,获得了包含变量之间依赖关系在内的不同节点之间的关联关系;最后将构建好的代码图数据作为输入,输入到图神经网络中进行表征学习。

Lu等人\cite{Lu2019ProgramCU}提出了一种用于程序分类的图网络模型GGANN(Gated Graph Attention Neural Networks)。该方法从代码中提取数据流与函数调用信息,将其融合到抽象语法树中,从而将代码构建为一个包含丰富信息的图结构表示FDA。在传统的GGNN模型上引入了注意力机制,用于获得图中每个节点的重要程度,进而获得更具有区分度的代码表征向量。

Brockschmidt等人\cite{Brockschmidt2018GenerativeCM}提出了一个生成代码模型,该模型利用部分生成程序的已知语义来指导生成过程。关键思想是在代码的抽象语法树上增加相应的边以构建代码图,然后扩充部分程序已获得图,之后使用图神经网络对部分程序的结构和数据流进行建模完成代码表征任务。这种表示有助于更好地指导生成过程的剩余部分。

Ben-Nun等人\cite{10.5555/3327144.3327276}提出了一种与语言以及平台无关的代码表征方法inst2vec。该方法首先使用编译器对代码进行编译,得到代码的中间表示。但由于该中间表示并没有包含代码之中的数据流信息以及控制流信息,因此该方法将数据流和控制流也融合到该中间表示中,进而构建了代码上下文流图。最后在所构建的图上使用循环神经网络进行建模,获得代码的表征向量。该向量在程序分类实验中的准确率取得了当时最好的效果。

Wang等人\cite{9054857}提出了一种称为流增强抽象语法树FA-AST(Flow-Augmented Abstract Syntax Tree)的程序图表示,考虑了仅仅使用代码的抽象语法树进行代码表征建模实际上仍然有代码结构上的缺失这一问题,构建了代码抽象语法树的图形表示FA-AST,通过将抽象语法树各个叶子结点相连构建出适合图神经网络处理的数据,然后应用两种不同类型的图神经网络GNN来检测克隆。

DeFreez等人\cite{10.1145/3236024.3236059}提出了一种学习嵌入方法Fun2Vec,将每个函数映射到连续向量空间中的向量,以便同义函数的向量非常接近。该方法采用随机游走算法,在程序的过程间控制流图上随机选择部分执行路径,捕获程序的层级结构,每条执行路径转换为一个标签序列,借助Word2Vec方法,把标签映射为连续实值向量,并通过神经网络训练函数的嵌入向量。

Kang等人\cite{Yang2021AGS}提出了一种针对代码补全问题的基于门控卷积网络模型CC-CCNN。该方法通过从代码表示中获得有效的代码特征,提出了一种分类机制,通过使用已知的父节点对节点的表示进行分类,并在模型中构建训练图。实验结果表明,模型在数据集中最多优于最先进的方法MRR最多9.2\%,ACC最多11.4\%。

上述方法均在图上进行代码的表征学习,力图充分提取代码中的语义信息。

(4)基于语法和语义融合的代码表征

基于语法与语义融合的模型,结合AST、DFG、CFG、Token序列,捕获程序的语法及语义结构信息。其中抽象语法树AST和Token序列反映了语法层面的信息,DFG、CFG反映了语义层面的信息。

Tufano等人\cite{Tufano2018DeepLS}采用四种不同的代码表征方法(即标识符、抽象语法树、字节码和控制流图)进行代码克隆检测,他们利用四种代码表示分别识别代码对的相似度,并计算平均值作为最终的相似度结果。

Saini等人\cite{10.1145/3236024.3236026}提出了一种代码克隆检测框架Oreo,该方法从程序的源代码中提取了包括被调用的外部方法的数量、变量的数量、语句的数量、循环的数量等24种度量,然后进一步从函数中抽取语义,并使用了基于哈希的方法进一步筛选,最后加入了深度学习的方法,将两个程序向量输入到孪生模型中来判断两个程序之间是否具有克隆关系。

Fang等人\cite{Fang2020FunctionalCC}结合抽象语法树,控制流图和调用图来学习代码特征,融合了语法和语义信息。首先,该方法从源码中分析出方法之间的调用图、每个方法的抽象语法树以及每个方法的控制流图;然后,用调用图将找出每个功能的AST集合。通过AST集合抽取功能的语法信息;通过调用图组成每个功能的控制流图;通过控制流图抽取功能的语义信息;最后,将抽取出来的语义信息送入前馈神经网络得到分类结果。

Hua等人\cite{Hua2020FCCAHC}提出了一种使用注意力的功能代码克隆检测器FCCA(Functional code clone detector using attention),结合标记、抽象语法树和控制流图三种方式实现检测目标。该方法通过保留多个代码特征,包括非结构化(以顺序令牌形式的代码)和结构化(以抽象语法树和控制流图形式的代码)信息,在混合代码表示的基础上进行代码克隆检测。将多个代码特征融合到混合表示中,该混合表示配备有注意力机制,有助于最终检测精度的重要代码部分和特征。

Dong等人\cite{9148302}提出了一种基于Token和AST的代码表征方式,提取数量特征如AST中的AST树的高度、节点数以及标记中操作数的个数、字符串的个数等作为神经网络的输入进行检测。

Feng等\cite{Feng2020CodeBERTAP}提出多模态的预训练模型,利用不同模态的信息互补作用,有效提升了模型的整体表征能力。CodeBERT基于文档和代码,在自然语言和程序语言双模态下,利用BERT进行预训练,提取自然语言和程序语言之间的语义连接,为下游任务提供通用表示向量。

Duan等人\cite{WOS:000680742600067}提出了一种无监督的程序代码表示学习技术DEEPBINDIFF,依靠代码语义信息和程序控制流信息生成基本块嵌入,并且采用k-HOP贪婪匹配算法利用基本块嵌入发现最优的相似性结果。通过大量二进制文件和真实的OpenSSL漏洞对原型进行评估,结果表明DEEPBINDIFF相比于最先进的工具,跨版本和交叉优化级别都更优。

Wu等人\cite{10.1145/3324884.3416562}提出了一种软件功能克隆检测方法SCDetector。该方法结合基于序列和基于图的方法。给定一个方法源代码,首先生成控制流图,然后应用社交网络中心性分析将图转换为某些语义标记(即具有图细节的标记)。最后,这些语义标记被馈送到Siamese网络中,以训练模型并使用它来检测代码克隆对。

总体而言,代码克隆检测是软件工程领域一项重要任务,如何对代码进行合适的表征是代码克隆检测的关键问题。代码表征学习决定了对源代码信息抽取程度的上限,决定了检测技术的预处理方法、模型设计、部署方式、运行效率,并会影响最终结果。面向代码克隆检测这一下游任务,代码表征学习研究存在以下不足:对代码结构信息语义信息利用不充分,特征表达不够完善;表征模型对数据集、模型结构和优化算法等多方面因素的要求高等问题,这些不足严重制约着代码克隆检测技术的发展。因此,研究人员一方面通过对源代码进行充分利用,提出多维源代码表征方法,从而提高代码克隆检测能力;另一方面,通过研究更先进的算法来提高表征模型的自动化和智能化程度,也是目前重要的发展趋势。

\section{研究内容}
\label{sec:Content}
本文主要围绕如何将源代码表征学习技术应用到代码克隆检测领域,通过不同维度对程序表征进行学习,并基于学习得到的语义特征进行克隆对的判定,充分发挥代码表征学习技术检测代码克隆的能力。针对现有代码表征学习方法存在的对代码结构信息和语义信息利用不充分的问题,本文提出面向代码克隆检测的多维源代码表征学习方法RLCCD ,旨在通过构建三个不同维度的代码表征模型,将源代码的语义信息表示为稠密低维实值向量,以在低维空间中高效计算实体和关系的语义联系,并通过特征融合得到多维特征,实现对代码信息的充分利用,以更加全面准确与智能化的方式提高代码克隆测试效率。

本文的主要工作包括:

(1)提出面向代码克隆检测的多维源代码表征学习方法RLCCD 

本文提出了一种面向代码克隆检测的多维源代码表征学习方法RLCCD,该框架主要针对代码表征关键步骤提出三个关键技术点,从Token序列、抽象语法树AST、程序依赖图PDG三种不同维度对代码特征表示进行优化,分别形成了基于预训练辅助模型的Token表征学习、基于子树划分的抽象语法树表征学习、基于图过滤的程序依赖图表征学习三种方法,然后通过特征融合将三种维度特征整合为一个多维特征,实现对代码信息的充分利用,以更加全面准确与智能化的方式提高代码克隆测试效率。 

(2)基于预训练辅助模型的Token表征学习

针对目前现有的基于Token的表征学习方法通常将代码表示为词汇单元,为了后续生成表征向量通常会将词汇单元规范化,丢失部分语法信息,出现在词汇表中不存在Token的难题,提出了一种基于预训练辅助模型的Token表征学习方法。该方法在模型训练之前,通过选取预训练辅助模型从代码语料库中学习基本单元的语法语义信息,以及这些单元之间的联系,最终给出一份单词-向量形式的词汇表,从而减少出现集外词问题的概率。本文在POJ104数据集上的消融实验评估表明,预训练辅助模型方法能够提高代码克隆检测准确率。

(3)基于子树划分的抽象语法树表征学习

针对现有的基于树的表征学习方法通常将抽象语法树转换为完整二叉树,可能破坏源代码原有的语法结构,增加AST高度,丢失长期上下文信息,削弱了神经模型捕捉更真实和复杂语义的能力,导致梯度消失的难题,提出了一种基于子树划分的抽象语法树表征学习方法。该方法将每个大型的AST分割成小语句树序列,并通过捕获语句的词法和句法知识将每一个语句树都编码成一个向量,在得到一个语句向量序列后,将语句向量序列输入树卷积神经网络中生成代码片段的结构向量表示。本文在POJ104数据集上的消融实验评估表明,子树划分方法能够有效提取结构特征,提高代码克隆检测准确率。

(4)基于图过滤的程序依赖图表征学习

针对现有的基于图的表征学习方法通常将程序表征为有向多重图,继而采用图匹配算法将图中的控制流和数据流编码为一个紧凑的语义特征矩阵,矩阵中每个元素都是高维系数特征向量,所消耗的时间、空间开销巨大的难题,提出了一种基于图过滤的程序依赖图表征学学习方法。该方法通过收集PDG的简单特征来过滤掉明显不可能为克隆的PDG对。具体的,根据PDG的节点个数、控制边数、执行边数、数据边数、声明节点数、函数调用数、传入参数、传出参数等代表特征进行过滤,从而减少候选PDG对规模。本文在POJ104数据集上的消融实验评估表明,图过滤方法能够有效减少时间、空间开销,提高代码克隆检测准确率。

(5)特征融合及实验验证

特征融合方法是指将不同来源或不同层次的特征进行组合,合并成一个比输入特征更具有判别能力的特征,该多维特征能够在低维空间中高效计算实体和关系的语义联系,提高特征的表达能力和分类效果,有利于下游代码克隆检测任务的学习。

本文选取了代码克隆检测领域常见的基准集POJ104进行实验验证,并与现有开源的SourcererCC\cite{7886988}、ASTNN\cite{8812062}、SCDetector\cite{10.1145/3324884.3416562}方法进行比较,主要通过召回率(Recall)、精确度(Precision)和准确率(Accuracy)三个指标评价实验结果,实验结果验证了RLCCD的可行性和有效性。

\section{论文结构}
\label{sec:Summary1}
本文总共分为六章,各个章节所包含的具体内容如下:

\textbf{第1章} \quad 绪论部分首先对本文的研究背景与意义进行了阐述,之后对代码克隆检测技术和代码表征学习技术的研究现状与趋势进行了分析总结,进而提出本文的主要研究内容,最后介绍了全文的组织结构。

\textbf{第2章}  \quad 分析了代码表征学习领域的关键技术挑战,基于此,提出了本文的面向代码克隆检测的多维源代码表征学习方法RLCCD,并对该技术的整体框架进行了介绍,进而根据所提框架简要论述了本文研究的关键技术,即基于预训练辅助模型的Token表征学习方法、基于子树划分的抽象语法树表征学习方法、基于图过滤的程序依赖图表征学习方法,最后进行本章小结。

\textbf{第3章}  \quad 介绍基于预训练辅助模型的Token表征学习方法的设计与实现。首先,分析其研究动机,即目前Token表征学习面临的集外词问题,继而提出基于预训练辅助模型的方法设计,详细介绍该方法的设计思路和具体实现,最后,通过消融实验验证,评估本章提出的预训练辅助模型的有效性。

\textbf{第4章}  \quad 介绍基于子树划分的抽象语法树表征学习方法设计与实现。首先,分析其研究动机,即目前抽象语法树表征学习面临的梯度消失问题,继而提出基于子树划分的方法设计,详细介绍该方法的设计思路和具体实现,最后,对该方法的有效性进行消融实验验证。

\textbf{第5章}  \quad 介绍基于图过滤的程序依赖图表征学习方法设计与实现。首先,分析其研究动机,即目前图表征学习面临的规模开销问题,继而提出基于图过滤机制的方法设计,详细介绍该方法的设计思路和具体实现,并给出了针对该方法有效性的消融实验验证。

\textbf{第6章}  \quad 介绍特征融合及本文研究框架RLCCD的实验验证。首先,针对特征融合的方法设计与具体实现进行了介绍。接着,对RLCCD框架的有效性进行了评估,通过与现有开源技术SourcererCC\cite{7886988}、ASTNN\cite{8812062}、SCDetector\cite{10.1145/3324884.3416562}进行实验对比,验证RLCCD方法的有效性。

\textbf{结论}  \quad 首先对全文的研究工作进行了总结,并讨论本文的主要贡献与创新之处,最后对下一步可开展的工作提出展望。
\chapter{多维源代码表征学习方法总体设计}
\label{chap:design}

本章首先阐述了当前代码表征学习面临的一些关键技术挑战,进而针对现有的问题提出面向代码克隆检测的多维源代码表征方法RLCCD。然后就该框架的总体架构和处理流程进行详细阐述,最后,给出了下游代码克隆检测任务的问题定义,针对RLCCD框架给出了形式化描述。

\section{代码表征学习面临的技术挑战}
\label{sec:challenges}
目前已有的代码克隆检测方法大多遵循以下思路:(1)首先对代码片段进行预处理;(2)对处理好的代码片段进行代码表征,将其转换为中间表征;(3)根据表征的方式不同计算不同代码片段之间的相似度,完成克隆检测任务。在代码克隆检测中,源代码表征方式决定了信息抽取的程度和粒度,进而影响了后续克隆检测的精度和效率。如何得到丰富且有效的源代码表征表示,是解决代码克隆检测任务的关键所在。从目前多种维度的代码表征方法来看,现有的代码表征方式存在以下技术挑战:

(1)Token维度代码表征存在集外词问题

基于Token的方法一般会利用词法分析器将源代码中的词汇单元Token划分出来,得到Token序列,并过滤掉无用的空格、注释、字符等,然后利用深度学习技术对其进行建模,生成具有丰富代码信息的表征向量,应用于下游代码任务。这类方法和自然语言处理(NLP)领域中常用来处理文本的方式很相似,产生一个规模巨大且稀疏的词汇表。但是,在大多数基于Token的代码克隆检测工具中,通常会将词法单元规范化,例如:将变量名用统一的标识符来代替。经过规范化Token产生的词汇表较小,导致模型学习能力有限,并且在训练过程中会出现未见过或未包含在词汇表中的词语。这些词语可能是用户自定义词、拼写错误、缩写、专有名词等。由于模型在训练阶段没有足够的信息来学习这些词语的表示,因此在实际应用中无法正确处理这些词语,从而导致模型的性能下降,这就是集外词(Out of vocabulary,简称OOV)问题。集外词问题会对模型的性能和泛化能力造成影响,严重限制了代码表征的有限性。

(2)树维度代码表征存在梯度消失问题

基于树的方法将代码通过语法解析转换成相应的抽象语法树,从而有效地表示代码的语法及其结构信息。与自然语言处理领域的长文本类似,当上下文序列很长的时候,基于树的神经网络模型容易出现梯度消失的问题,即梯度在训练过程中变得越来越小,特别是当树非常深的时候,模型会面临梯度消失问题。目前大多数基于树的代码克隆检测方法为了简化或者提高效率,通常会将生成的抽象语法树转换为完整的二叉树,在转换的过程中,不仅破坏了源代码原有的语法结构,也会增加树的高度,进一步削弱模型捕捉复杂语义的能力,导致检测性能下降。

(3)图维度代码表征存在规模开销问题

基于图的方法会将源代码表征为数据流图或者控制流图,数据流图代表了源代码中数据的走向,控制流图代表了代码中语句执行时的跳转流向。大多数基于图的代码克隆检测工具任务的核心是将图中的每个节点映射到一个低维、稠密的特征向量中,并将这些特征编码为特征矩阵,这一步通常需要大量空间开销。同时子图匹配算法是NP完全问题,计算成本过长,时间复杂度很高,因此图维度代码表征学习会存在算法计算开销大,可扩展性不好,检测结果召回率低等问题。

(4)代码表征存在信息利用不充分问题

虽然目前代码表征在Token、树、图等多种维度的研究已经取得了一定的进展,但还存在信息利用不充分的问题。代码不仅仅具有文本自然性,同时具有结构信息、语义信息。在现有的表征粒度中,Token维度的代码表征通常只关注文本自然性,抽象语法树可以捕获程序的语法结构和模式,程序依赖图可以表达程序的部分语义信息,只使用单个特征来表示代码是远远不够的,很难覆盖所有信息,因此存在信息利用不充分,特征表达不完善的问题。

\section{RLCCD研究方案}
\label{sec:Framework}

\subsection{研究思路及总体框架}
\label{subsec:Ideas}
针对\ref{sec:challenges}节提出的四个技术挑战,本文提出了如图\ref{fig:thinking}所示的研究思路。

具体地,本文主要针对代码表征学习的三个维度展开研究工作:(1) 针对Token序列特征挖掘,提出预训练增强辅助模型提取属性特征,从而解决传统基于Token序列的方法存在的集外词问题;(2)针对抽象语法树AST特征挖掘,提出子树划分的改进方法提取结构特征,从而解决传统基于抽象语法树的方法存在的梯度消失问题;(3)针对程序依赖图PDG特征挖掘,提出过滤机制提取语义特征,通过收集PDG的简单特征来过滤输入神经网络模型的输入,从而解决传统基于程序依赖图的方法存在的规模开销问题。

同时,考虑到经由不同表征方式处理所得到的信息通常具有互补性,且不同维度的特征都是代码表示的平行语料,具有信息等价性,因此,本文提出基于多模态学习的特征融合方法,通过融合多个代码特征,包括非结构化(顺序Token形式的代码)和结构化(抽象语法树和程序依赖图形式的代码)信息,从多维数据中学到更好的特征表示,从而有利于提高下游代码克隆检测任务的检测精度。

\begin{figure}[H]
    \centering
    \includegraphics[width=0.75\textwidth]{figures/thinking}
    \caption{研究思路}
    \label{fig:thinking}
\end{figure}

本文基于上述研究思路,设计了面向代码克隆检测的多维源代码表征方法RLCCD,框架如图\ref{fig:framework}所示。由图\ref{fig:framework}可见,本文提出的基本框架与\ref{subsec:Code clone detection}节提出的代码克隆检测的处理流程基本一致,并主要通过三个维度对代码表征学习环节进行改进,然后对三个维度得到的特征向量进行特征融合,得到的混合特征应用到下游代码克隆检测任务中。

\begin{figure}[H]
    \centering
    \includegraphics[width=0.75\textwidth]{figures/framework}
    \caption{RLCCD 总体框架}
    \label{fig:framework}
\end{figure}

\subsection{代码预处理}
\label{subsec:Preprocess}
代码预处理的目标是生成源代码片段对应的词法单元Token序列、抽象语法树AST和程序依赖图PDG,主要包含2个流程:代码标准化、生成中间表示。

(1)代码标准化

代码标准化的任务是去除与源代码无关的信息。首先是删除源代码片段中的注释、空行以及特殊符号,包括单行注释、多行注释、引入标准库的宏符号“\#”、无关符号等。其次,由于代码本身是一段包含丰富信息的文本,开发者通常会通过个人命名习惯对常量等标识符进行命名,这些私人信息并没有大多实际意义,反而会降低后续代码处理的精确率。因此,本文定义了一定的转换规则,将代码中的某些标识符转换为对应的标记,在最大限度地保留原有重要信息的同时,减少缺失关键语义联系。

(2)生成中间表示

基于标准化后的代码片段生成对应的中间表示:Token序列、抽象语法树AST和程序依赖图PDG。其中,词法单元Token序列可以通过词法分析器得到,词法分析器能够按照预定的语法规则将代码中的字符串分割为一个个词汇单元,这些词汇单元包含代码标准化处理后的标识符;抽象语法树AST以树状的形式抽象描述了程序语句的语法结构信息,生成抽象语法树时需要对源代码文本进行词法分析,然后依据语法规则分析整合Token,得到树型结构;程序依赖图PDG能够表示源代码的控制依赖,数据依赖赖等关系,是一种带有标记的有向多重图。生成程序依赖图时,需要对程序进行语法分析,然后分析程序中变量的关联关系,根据这种关联关系描绘程序的数据依赖和控制依赖关系,形成图的描述。

\subsection{多维源代码表征学习}
\label{subsec:Representation}
RLCCD框架的核心步骤是源代码表征学习,其目标是学习能够表示代码片段的连续向量,表现程序理解的认知层次,获取程序的语法、语义信息,创建程序更高抽象层次上的表示,它决定着对源代码信息抽取程度的上限,决定着检测方法的预处理方式、模型设计、部署方式、运行效率,并影响后续代码克隆检测任务所能检测的精度。本文提出的多维源代码表征学习方法包括Token序列、抽象语法树AST、程序依赖图PDG三种不同维度。下面详细介绍研究方案并分析其优化改进。

(1)词法单元Token

传统的基于Token的表征学习方法通常将代码表示为词汇单元,为了后续生成表征向量通常会将词汇单元规范化,丢失部分语法信息,出现在词汇表中不存在Token的难题。针对这样的问题,本文提出了一种预训练辅助模型的改进方法提取属性特征。具体来说,通过选取预训练辅助模型从代码语料库中学习基本单元的语法语义信息,以及这些单元之间的联系,最终给出一份单词-向量形式的词汇表,从而减少出现集外词问题的概率。

(2)抽象语法树AST

抽象语法树中包含了代码片段的结构信息,然而,传统的基于树的代码表征学习方法通常将抽象语法树转换为完整二叉树,可能破坏源代码原有的语法结构,增加AST高度,丢失长期上下文信息,削弱了神经网络模型捕捉更真实和复杂语义的能力,导致梯度消失的难题。针对这样的问题,本文提出子树划分的改进方法提取结构特征。具体来说,将每个大型的抽象语法树分解为小语句树序列,并通过捕获语句的词法和句法知识将每一个语句树都编码成一个向量。在得到一个语句向量序列后,将语句向量序列输入网络中生成代码片段的结构向量表示。这种细粒度的处理使得模型可以很好处理很深的抽象语法树,解决梯度消失问题。

(3)程序依赖图PDG

程序依赖图中包含代码片段的控制依赖,数据依赖赖等语义关系,然而,传统的基于图的代码表征学习方法通常采用图匹配算法将图中的控制流和数据流编码为一个紧凑的语义特征矩阵,矩阵中每个元素都是高维系数特征向量,面临消耗的时间、空间开销巨大的难题。针对这样的问题,本文提出了图过滤机制的改进方法提取语义特征。具体来说,根据PDG的节点个数、控制边数、执行边数、数据边数、声明节点数、函数调用数、传入参数、传出参数等代表特征进行过滤,减少模型的输入规模。

(4)特征融合方法

特征融合的目标是将提取到的属性特性、结构特征、语义特征合并,得到一个更能代表代码信息的多维特征,更具有判别能力。具体来说,通过三个不同维度得到属性特性、结构特征、语义特征,然后将三个特征映射到相同的特征空间内,最简单的例子是对多个代码特征进行串联,除此之外,还可以使用神经网络、概率图模型等融合方法,将向量融合作为一个混合多维表征,包括非结构化(顺序Token形式的代码)和结构化(抽象语法树和程序依赖图形式的代码)信息。该多维特征能够在低维空间中高效计算实体和关系的语义联系,挖掘代码节点间更全面、层次更深的关系信息,从而提高后续下游的代码克隆检测任务的准确率。

\subsection{克隆检测任务实现}
\label{subsec:Clone detection}
克隆检测任务实现的核心任务是判断两个代码片段是否是真克隆对。经过\ref{subsec:Preprocess}节代码预处理和\ref{subsec:Representation}节多维源代码表征学习两个步骤,可以得到对应的混合特征向量表示作为输入,然后计算这两个向量之间的相似性判断是否存在代码克隆。目前,常见的计算向量相似性的方法包括计算距离度量、相似性度量两种,向量的距离越近相似度越大。一些研究倾向于把程序表征为向量形式,使用余弦相似度、Jaccard相似度、欧几里得距离、汉明距离、曼哈顿距离等评估指标计算向量之间的相似度,当相似度大于某个固定阈值时则认为存在代码克隆。具体的,本文通过输入混合特征训练分类模型,以最小化模型损失为目标,完成代码克隆检测任务。

\section{RLCCD定义描述}
\label{sec:RLCCD flow}

为了更形象地描述代码克隆检测问题,本节首先给出示例源代码片段\ref{fig:code},其中,图\ref{fig:code1}和图\ref{fig:code2}是两个简单的代码片段,函数的主要功能为:计算$Data$数组的元素之和,其中图\ref{fig:code1}的第10-13行采用for循环,图\ref{fig:code2}的第9-13行采用while循环,这两个代码片段属于真克隆对。
\begin{figure}[H] 
  \centering  %居中
  \subfigure[C语言代码片段$C_{a}$]{   %第一张子图
      \centering    %子图居中
      \includegraphics[width=0.45\textwidth]{figures/code1}
      \label{fig:code1} %引用标签
  }
  \subfigure[C语言代码片段$C_{b}$]{ %第二张子图
      \centering    %子图居中
      \includegraphics[width=0.45\textwidth]{figures/code2}
      \label{fig:code2} %引用标签
  }
  \caption{C语言代码片段示例}    %大图名称
  \label{fig:code}    %图片引用标记
\end{figure}
% \lstset{language=C}
% \begin{lstlisting}
%     /* Count the sum of elements in an array */
%     #include <stdio.h>
    
%     int main()
%     {
%         int i = 0;//parameter one
%         int data[5]={1,2,3,4,5};//parameter two
%         int result = 0;//parameter three
        
%         for(;i<5;i++)
%         {
%             result += data[i];
%         }
%         printf("%d",result); 
%         return 0;
%     }
% \end{lstlisting}

% \lstset{language=C}
% \begin{lstlisting}
%     /* Count the sum of elements in an array */
%     #include <stdio.h>
    
%     int main()
%     {
%         int i = 0;//parameter one
%         int data[5]={1,2,3,4,5};//parameter two
%         int result = 0;//parameter three
%         while(i<5)
%         {
%             result += data[i];
%             i++;
%         }
%         printf("%d",result); 
%         return 0;
%     }
    
% \end{lstlisting}

给定两个代码片段$C_{a},C_{b}$,使用一个三元组$(C_{a},C_{b},y_{ab})$的形式来表示一对代码片段,其中$y_{ab}$表示标签。如果$(C_{a},C_{b})$是一个克隆对,那么$y_{ab}$为1,否则$y_{ab}$为0。从$n$个代码片段中构建一个带有标签的训练集$\left\{(C_{a},C_{b},y_{ab})|a,b \in n,a<b\right\}$,本文的目的是训练一个深度学习模型来学习一个可以把代码$C$映射成一个特征向量$V$的函数$f$。对于任意代码片段,计算出他们的相似度分数$S_{ab} = f(C_{a},C_{b})$并进行分类,使其分类结果尽可能接近已知的标签$y_{ab}$。在预测阶段,为了推断出两个代码片段是否是克隆对,在真假克隆对之间设置了一个阈值$k$,如果预测的相似度分数大于$k$值,那么认为两个代码片段是真克隆对,否则,认为它们是假克隆对。其中,训练一个深度学习模型将代码$C$映射成一个特征向量$V$的过程,可以视为代码表征学习的数值化过程,也是本文研究的重点。

面向代码克隆检测的多维源代码表征学习方法RLCCD架构如图\ref{fig:flow}所示,采用Siamese架构。RLCCD方法的输入是两个代码片段$C_{a},C_{b}$,输出是一个0或1的标签,1表示这两个代码片段是真克隆对,0表示是假克隆对。Siamese架构采用两个不同的输入,通过两个具有共享权值的相似子网络,输出一个编码来计算两个输入之间的相似度。整个方法的执行包括四个步骤,分别是:
\begin{figure}[H]
    \centering
    \includegraphics[width=0.95\textwidth]{figures/flow}
    \caption{面向代码克隆检测的多维源代码表征学习方法RLCCD架构图}
    \label{fig:flow}
\end{figure}

\textbf{代码预处理}:RLCCD接受两个代码片段$C_{a},C_{b}$作为输入,通过代码预处理阶段,得到对应的中间表示:词法单元Token序列、抽象语法树AST、程序依赖图PDG。

\textbf{代码表征}:代码表征是本文的核心步骤,输入是代码片段$C_{a},C_{b}$对应的中间表示,输出是对应的表征向量,分别记为代码片段$C_{a}$的属性特征$V_{a}^{Token}$、结构特征$V_{a}^{AST}$、语义特征$V_{a}^{PDG}$,代码片段$C_{b}$的属性特征$V_{b}^{Token}$、结构特征$V_{b}^{AST}$、语义特征$V_{b}^{PDG}$。

\textbf{特征融合}:特征融合阶段的输入是代码片段$C_{a},C_{b}$的特征向量,输出是对应的混合向量$V_{a},V_{b}$。本文将提取到的属性特性、结构特征、语义特征合并,得到一个更能代表代码信息的多维特征。我们用公式\ref{e4}表示特征融合。

\begin{equation}\label{e4}
    \setlength{\abovedisplayskip}{1pt}
    \mathrm{V}=W_{dt}\left[v^{\text{Token}} \bullet v^{\text{AST}} \bullet v^{\text{PDG}}\right]+b_{dt}
\end{equation}

其中$\bullet$表示特征融合方法,$V$表示最终的多维混合代码表示。$W_{dt}$在生成的多维混合表示中平衡属性特征$V^{Token}$、结构特征$V^{AST}$、语义特征$V^{PDG}$的组成,而$b_{dt}$在训练模型时使模型偏向最终收敛。

\textbf{克隆检测}:经过三个步骤,可以得到代码片段对应的向量表示$V_{a},V_{b}$。由于代码克隆检测问题是一个二分类问题,即给定两个代码片段,需要输出0或1,0表示它们之间不相似,1表示相似。因此通过公式\ref{e1}计算代码片段$C_{a}$对应的多维表征$V_{a}$与代码片段$C_{b}$对应的多维表征$V_{b}$之间的距离$d$,并将向量距离$d$映射到0$\sim$1之间,将输出值作为两个代码片段的相似度$S_{ab}$。
\begin{equation}\label{e1}
    \begin{split}
    d &= \left|V_{a}-V_{b}\right| \\
    \mathrm{S_{ab}} &=\operatorname{sigmoid}\left(d\right) \in[0,1]
    \end{split}
\end{equation}

并将损失函数定义为二元交叉熵,如公式\ref{e2}所示,训练模型的目标是最小化损失。
\begin{equation}\label{e2}
    J(\Theta, S_{ab}, y_{ab})=\sum(-(y_{ab} \cdot \log (S_{ab})+(1-y_{ab}) \cdot \log (1-S_{ab})))
\end{equation}

其中,$y_{ab}$表示两个代码片段的真实标签。

当所有参数都设置为最优化后,模型被存储起来。在预测阶段,通过公式\ref{e3}得到预测值。
\begin{equation}\label{e3}
    \text { prediction }=\left\{\begin{array}{ll}
        1, & S_{ab}>k \\
        0, & S_{ab} \leq k
        \end{array},\right.
\end{equation}

\section{本章小结}
\label{sec:Summary2}
本章首先给出了代码克隆检测问题的定义,然后分析了源代码表征学习在代码克隆检测过程中所面临的关键技术挑战,主要表现为Token集外词问题、树梯度消失问题、图规模开销、单个表征维度对代码信息利用率低问题。针对四个问题,本文提出了面向代码克隆检测的多维源代码表征学习方法RLCCD,在介绍了其整体框架后,对其中的关键技术点进行了简要的论述,最后详细描述了RLCCD的处理流程。

\chapter{基于预训练辅助模型的Token表征学习}
\label{chap:Token}
本章主要对本文提出的基于预训练辅助模型的Token表征学习方法进行详细介绍,首先介绍其研究动机,接着阐述其方法设计,以及具体的实现过程,最后介绍实验验证过程和结果。

\section{研究动机}
\label{sec:Motivation}

基于Token的代码表征方法本质上就是将源代码转换为一系列词法单元Token组成的序列,并对Token序列进行代码分析。类似于自然语言技术处理文本,由于源代码中存在大量用户自己定义的标识符,不同用户的命名习惯不同,在对源代码的词法单元建模时,会产生一个规模巨大且稀疏的代码词表。该代码词表的规模会直接影响代码分析任务的效率,因此现有方法大多都对Token进行规范化,比如将变量名用统一的标识符来代替,从而降低代码词表的规模。但是后续神经网络模型训练过程中,当出现某个词汇在代码词表中没有出现过,那么神经网络模型就无法对齐建模,即出现了在代码词表中不存在的Token,集外词(Out-of-vocabulary,简称OOV)问题。有研究\cite{RJXB202205011}发现,针对代码表征中的集外词问题,在经典BigcloneBench\cite{7332459}数据集中OOV比率高达62.68\%,在OJClone数据集\cite{WOS:000485474201046}中OOV比率达到了16.82\%。

近期,研究人员在大规模语料库上预训练各种语言模型,在解决各种自然语言处理任务方面取得了良好进展\cite{zhao2023survey}。在表征学习领域,也有基于大规模预训练模型提升代码表征能力的方法被提出。InferCode\cite{9402028}将自然语言处理中的自监督学习思想引入到代码的抽象语法树的表示中, 通过预测从AST上下文中自动识别的子树来训练代码表征,并且使用AST的子树作为训练标签,从而无需任何人工标记工作。该预训练InferCode模型可以应用于下游的无监督学习,例如代码聚类、代码克隆检测、跨语言代码搜索等。GraphCodeBERT\cite{guo2021graphcodebert}方法提出了一个基于数据流的代码表征预训练模型,与抽象语法树不同,数据流包含代码变量间“值从哪里来”的语义特征,并不会带来深层次不必要的复杂信息,使用该特征可以更有效的生成代码表征。然而这些预训练模型通常存在参数规模庞大, 训练及使用代价大的问题。

因此,针对集外词问题,本文提出了一种轻量级的基于预训练辅助模型的Token表征学习方法,该方法在提升代码表征能力的同时,并不会引入过多参数,造成训练代价大的问题。

\section{Token表征方法设计}
\label{sec:Token}
本节将主要介绍基于预训练辅助模型的Token表征学习方法设计与实现,首先介绍该方法的整体框架,并分别从预训练辅助模型、Token表征学习两方面介绍具体设计及实现。

\subsection{框架概述}
\label{subsec:Overview}
本文提出的基于预训练辅助模型的Token表征学习方法的整体框架如图所示。

\subsection{预训练辅助模型}
\label{subsec:Model}

Word2vec模型 多次迭代维护词汇表,从而减少OOV问题

在模型训练之前,通过选取合适的模型从代码语料库中学习基本单元的语法语义信息,以及这些单元之间的联系,最终给出一份单词-向量形式的词汇表,从而减少出现集外词问题的概率。具体的,第一次迭代选择相邻的两个token组合为一个单元,查找出最频繁的token组合确定为一个组合单元,并将组合单元更新到词汇表中,然后二次迭代,每次迭代在原来的基本单元上再组合一个新的邻近token作为新的判断单元,每次迭代都会更新词汇表。在得到词汇表之后,根据词汇表获取每个单元的向量表示。

\subsection{Token表征学习}
\label{subsec:Token}
采用BiLST模型

将token序列对应的向量输入到Bi LSTM网络中,经过Bi LSTM学习哪些信息应该被记住哪些信息应该被遗忘,最终得到每个基本单元包含语义信息和上下文信息的向量表示。Bi LSTM由一个正向的LSTM和一个反向的LSTM组成,主要思想是通过把序列向前、向后分别输入给两个独立的递归网络,这两个子网络连接到一个输出层,在每个词的输出部分把两个子网络的输出信息进行整合,这样网络就同时拥有了序列中每个词的过去时刻信息和未来时刻信息。Bi LSTM可以捕获到序列前后的关系依赖,将代码片段的Token序列转换为可相互比较的向量。

\section{实验验证}
\label{sec:Experiment}
为了验证基于预训练辅助模型的Token表征方法的有效性,本节展开实验验证。首先,介绍了实验整体的环境配置、数据集,以及实验评估;接着,对基于预训练模型的Token表征方法进行了消融实验。

\subsection{实验环境}
\label{subsec:Environment}
本章的实验验证均运行于Linux系统下,其系统硬件配置如表\ref{tab:environment}所示。

\begin{table}[H]
  \centering
  \caption{实验环境配置} 
  \label{tab:environment}
  \begin{tabular*}{0.6\textwidth}{@{\extracolsep{\fill}}cc}
  \toprule
    环境			&配置		\\
  \midrule
    操作系统		&Ubuntu 20.04 \\
    处理器			&Intel Core i9-12900KF × 24 \\
    内存			  &31.1G \\
    显卡			  &NVIDIA  \\
    磁盘			  &1TB \\
  \bottomrule
  \end{tabular*}
\end{table}

\subsection{实验数据集}
\label{subsec:Dataset}
为了验证预训练辅助模型在Token层面上表征学习的有效性,本文面向代码克隆检测任务对预训练辅助模型进行分析和评估,选取的实验数据为POJ104数据集。如表\ref{tab:dataset}所示,POJ104数据集是一个基于C语言所构建的大型数据集。OJ系统是一个以编程教学为目的公开评判系统,共存在104个编程问题,针对每个编程问题,学生们通过在线提交自己的代码来尝试解决,同时OJ系统将自动判断提交源代码的正确性和有效性。对于OJ系统中同一个编程问题来说,其所有正确提交的代码都为克隆代码,对于不同的编程问题所提交的代码,即为非克隆代码。POJ104数据集针对每一个编程问题,均提供500个学生提交源代码,即共有52000个样本。

\begin{table}[H]
  \centering
  \caption{POJ104数据集} 
  \label{tab:dataset}
  \begin{tabular*}{0.6\textwidth}{@{\extracolsep{\fill}}cc}
  \toprule
    代码			&属性		\\
  \midrule
    Dataset			&POJ104数据集 \\
    Language    &C \\
    Program			&52000 \\
    Classes			&104 \\
    Max tokens			&8737 \\
    Avg tokens			&245 \\
  \bottomrule
  \end{tabular*}
\end{table}

得到POJ104数据集后,本文首先对数据集进行初步筛选,去掉其中包含乱码的样本,共得到51485个源代码样本。然后对源代码进行预处理,删除样本中包含的空白行和注释等多余代码,并将数据集保存到一个program.pkl文件中,program.pkl文件中一共包含51485行×3列的数据集,每一行数据代表一个源代码样本,第一列为源代码id,第二列保存源代码样本code,第三列为源代码的标签label,即属于哪一个编程问题。接着,本文随机两两组合同一标签label的源代码,组成5200个真克隆对,随机组合不同标签的源代码组成44800个假克隆对,一共给包含50000个克隆对,并将其保存到oj\_clone\_ids.pkl文件中,oj\_clone\_ids.pkl文件中一共包含50000行×3列的数据集,每一行数据代表一个克隆对样本,第一列为源代码id1,第二列为源代码id2,第三列为克隆对的标签label,真克隆对标签为1,假克隆对标签为0。最后,依据随机种子将数据集按照3:1:1划分为训练集、测试集、验证集,其中的正负样本数如下表\ref{tab:ClonePairs}所示。

\begin{table}[H]
  \centering
  \caption{本文预处理后的POJ104数据集正负样本数} 
  \label{tab:ClonePairs}
  \begin{tabular*}{0.8\textwidth}{@{\extracolsep{\fill}}cccc}
  \toprule
    数据集			&真克隆对		&假克隆对		&克隆对数 \\
  \midrule
    训练集train			&3162	  &26838		&30000 \\
    测试集test			&1022		&8978		  &10000 \\
    验证集dev			  &1016		&8984		  &10000 \\
    总计            &5200	  &44800	  &50000 \\
  \bottomrule
  \end{tabular*}
\end{table}

\subsection{评估指标}
\label{subsec:Index}
代码克隆检测问题是二分类问题,因此本文采用准确率(Accuracy)、精确率(Precision)、召回率(Recall)、F1值四个评估指标来度量实验结果,其中使用了混淆矩阵中的TP、FN、FP、TN,如表\ref{tab:ConfusionMatrix}所示。

\begin{table}[H]
  \centering
  \caption{分类问题的混淆矩阵} 
  \label{tab:ConfusionMatrix}
  \begin{tabular*}{0.7\textwidth}{@{\extracolsep{\fill}}ccc}
  \toprule
  \multirow{2}{*}{实际值} & \multicolumn{2}{c}{预测值} \\
  \multirow{2}{*}{} & 正样本(P) & 负样本(N) \\
  \midrule
    正样本(P)			&TP	  &FN		 \\
    负样本(N)			&FP		&TN		 \\
  \bottomrule
  \end{tabular*}
\end{table}

其中,混淆矩阵中的真阳性、假阳性、真阴性、假阴性代表的含义如下:

真阳性(True Positive, TP):样本实际为正样本,并且被模型预测为正样本,即实际上标记为真克隆对并且被检测出来为真克隆对的代码对。
 
假阳性(False Positive, FP):样本实际为负样本,但是被模型预测为正样本,即实际上标记为假克隆对但是被检测出来为真克隆对的代码对。
 
假阴性(False Negative, FN):样本实际为正样本,但是被模型预测为负样本,即实际上标记为真克隆对但是被检测出来为假克隆对的代码对。
 
真阴性(True Negative, TN):样本实际为负样本,并且被模型预测为负样本,即实际上标记为假克隆对并且被检测出来为假克隆对的代码对。

准确率(Accuracy)表示预测为正的样本中真实值为正的比率,计算公式如\ref{e5}所示:
\begin{equation}\label{e5}
  Accuracy = \frac{TP+TN}{TP+FN+FP+TN} 
\end{equation}

精确率(Precision)表示正确检测到的代码克隆数量占全部预测为代码克隆的比例,计算公式如\ref{e6}所示:
\begin{equation}\label{e6}
  Precision = \frac{TP}{TP+FP} 
\end{equation}

召回率(Recall)表示正确检测到的代码克隆数量占总体实际代码克隆数量的比例,计算公式如\ref{e7}所示:
\begin{equation}\label{e7}
  Recall = \frac{TP}{TP+FN} 
\end{equation}

精确率(Precision)和召回率(Recall)指标有时候会出现的矛盾的情况,这样就需要综合考虑两者的表现,最常见的方法就是F1,精确率和召回率的加权调和平均,用于评价分类模型的好坏。计算公式如\ref{e8}。
\begin{equation}\label{e8}
  F1 = \frac{2*Precision*Recall}{Precision+Recall} 
\end{equation}

\subsection{预训练辅助模型消融实验结果}
消融对比实验:体现改进的辅助模型的有效性,如图\ref{tab:category}
基于Token的Bi LSTM
基于Token的+预训练辅助模型的Bi LSTM

\begin{table}[H]
  \centering
  \caption{预训练辅助模型实验结果} 
  \label{tab:category}
  \begin{tabular*}{0.8\textwidth}{@{\extracolsep{\fill}}cccc}
  \toprule
    对比			&P		&R		&F1 \\
  \midrule
    基于Token的Bi LSTM			&0.xx	&0.xx		&0.xx \\
    基于Token的+预训练辅助模型的Bi LSTM			&0.xx		&0.xx		&0.xx \\
  \bottomrule
  \end{tabular*}
\end{table}

\section{本章小结}
\label{sec:Summary3}
本章主要对RLCCD中基于预训练辅助模型的Token表征学习方法的设计与实现进行详细阐述。首先介绍了Token维度的研究动机,其次介绍了Token表征学习的方法设计,具体论述了其整体框架、预训练辅助模型、表征学习,接着开展实验验证,结果表明了此方法的有效性和模型的准确性。

\chapter{基于子树划分的抽象语法树表征学习}
\label{chap:AST}
本章主要对本文提出的基于子树划分的抽象语法树表征学习方法进行详细介绍,首先介绍其研究动机,其次阐述其具体方法设计与实现,最后进行实验验证。

\section{研究动机}
\label{sec:ASTMotivation}
基于Token的代码表征通常将代码视为自然语言文本,根据程序员的开发风格不同,代码的命名方式、上下文组织方式也不一样,因此在代码表征过程中通常会产生很多噪声,并且遗漏源代码的结构信息。为了提高代码表征能力,研究人员提出了基于抽象语法树的代码表征方法。抽象语法树是源代码语法结构的一种抽象表现形式,以操作数、操作符、声明节点、语句节点等作为树节点,以树的形式包含了源代码中的语法信息和语法结构。基于树的方法利用了树本身的结构化特征,利用深度神经网络对树进行建模得到其向量表示,根据该特征向量完成下游代码任务,在一定程度上可以消除源代码本身的噪声,有效地提取程序的结构信息。

基于树的代码表征方法存在两种主要限制:

(1)树转化导致高度增加:由于抽象语法树的大小和深度对神经网络性能有显著影响,因此现有的基于树的方法通常将抽象语法树转化为二叉树,通过将父节点的第三个或者更多子节点移动到新的子树中进行简化。然而,在转换的过程中会改变源代码原有的语义,从而难以捕捉远程依赖关系,甚至丢失一些上下文信息。有研究发现\cite{Allamanis2017LearningTR},在抽象语法树中,具有高度关联的两个节点可能相距甚远,例如:函数调用的形参节点与实参节点存在数据依赖关系,但两者可能在转化过程中划分到不同的子树中,降低对程序的结构语法捕获能力。因此,由于转化会增加抽象语法树高度,从而加重了梯度消失问题,削弱神经模型捕捉更真实和复杂语义的能力。

(2) 神经网络梯度消失:在使用神经网络对抽象语法树建模的过程中,梯度是通过树型拓扑结构的反向传播来计算的。由于源程序结构的复杂性,因此抽象语法树通常规模过大、节点很深,此时会出现梯度消失问题。具体地,在反向传播过程中,神经网络根据设定好的损失函数指导权重值的更新优化。而梯度(即损失函数对模型参数的导数)经过多层网络传递时,如果激活函数的导数接近于0,参数更新就会变得不稳定,导致模型发散或者训练不收敛,影响神经网络模型的训练效率和稳定性。

因此,针对上述问题,本文提出了一种基于子树划分的抽象语法树表征方法,该方法将大型抽象语法树分割为小型语句树序列,在减小抽象语法树大小和深度的同时,通过捕获语句树的特征,提高树维度的代码表征能力。

\section{AST表征方法设计}
\label{sec:AST}
本节将介绍基于子树划分的抽象语法树表征学习方法的详细设计,首先介绍该方法的整体框架,并从子树划分、抽象语法树表征学习两方面介绍具体设计。 

\subsection{框架概述}
\label{subsec:ASTOverview}
本文提出的基于子树划分的抽象语法树表征学习方法整体框架如图\ref{fig:astframework}所示。该框架的输入是代码片段对应的抽象语法树,输出是对应的结构特征向量,主要包括子树划分、树表征两个阶段。

\begin{figure}[H]
  \centering
  \includegraphics[width=0.95\textwidth]{figures/astframework.png}
  \caption{基于子树划分的抽象语法树表征学习框架}\label{fig:astframework}
\end{figure}

首先,子树划分阶段以代码片段对应的抽象语法树作为训练数据,以抽象语法树的声明节点和语句节点作为切割粒度,设计一个基于先序遍历的子树划分算法,将抽象语法树切分成若干子树,得到对应的子树序列。

其次,树表征阶段以子树序列作为输入,输出代码片段对应的结构特征向量。该阶段分为两个部分,首先针对每一个子树,构建一个基于树的卷积神经网络将子树编码成向量,实现对细粒度语义的捕捉;其次,构建一个基于双向门控循环单元(BiGRU)的神经网络模型,通过最大池化层对子树特征进行压缩,输出一个固定长度的密集向量用来表示代码的结构特征。需要注意的是,在树表征阶段,本文使用两种神经网络模型,前者在语句粒度上对代码信息进行特征提取,后者引入在GRU模型的基础上引入双向结构,从而更好地捕捉序列数据地双向依赖关系。

在上述框架中,本文的创新点主要体现在子树划分阶段的子树划分算法、树表征阶段的树卷积、BiGRU模型设计两方面,下面将围绕这两个创新点来阐述本文的方法。

\subsection{子树划分设计}
\label{subsec:ASTPreModel}

抽象语法树是源代码语法结构的一种抽象表现形式,以树的形式包含了源代码中的语法信息和语法结构,不同的节点类型代表了源代码中的不同元素。其中,声明节点和语句节点是两种关键的节点类型。

声明节点用来定义代码中变量、函数、类等元素的声明,不仅包含了声明的标识符名称,还包含了该标识符的类型信息、作用域以及可能的初始值等。例如,在编译时,编译器可以通过检查声明节点来确保变量在使用前已经被正确声明,并且其类型与使用场景相匹配。

语句节点则用来定义代码的执行语句,如赋值语句、条件语句、循环语句、函数调用等。在AST中,语句节点通常包含了语句的类型、操作数以及可能的控制流信息。例如,在编译条件语句时,编译器可以通过分析语句节点来确定条件表达式的求值结果,并据此决定程序的执行路径。

抽象语法树存在因为树规模过大、高度过深导致的梯度消失问题,本文设计了一个基于先序遍历的子树划分算法,通过在语句粒度上对AST进行划分,在减少树高度的同时,实现对细粒度语义的捕捉。下面首先对本文提出的子树划分算法涉及到的基本元素进行定义,接着详细介绍整个算法流程。

\textbf{定义4.1.}抽象语法树语句节点集合:对于一个代码片段的AST,假设该AST的节点集合为$T = \left\{T_1,T_2,\ldots,T_n\right\}$,影响子树划分标准的语句节点包括:块语句、FOR循环语句、While循环语句、条件语句、返回语句,因此本文定义这些不同类型语句节点构成集合$S = \left\{BlockStatement,IfStatement,WhileStatement,ForStatement,\notag \right.\\\left.ReturnStatement\right\}$。

具体来说,子树划分算法需要寻找一个子树节点集合,表示为$SubT = \left\{SubT_i |\notag \right.\\\left. SubT \in T, SubT \in S,i=1,2,\ldots,k\left(k<n\right)\right\}$,其中集合$SubT$通过先序遍历AST节点获得,$n$表示AST中节点的总数量。

子树划分$split\_AST$算法的伪代码如\ref{alg2}所示,该算法有两个输入:根节点$Root\_Node$、语句节点集合$S$,输出为子树序列$SubTrees$。该算法的目的是按照先序遍历抽象语法树,根据给定的语句节点集合$S$,从根节点$Root\_Node$开始,判断每个子节点$child\_node$,是否需要将其作为新的子树进行切分(算法第4行),如果子节点属于语句节点集合,则直接加入子树节点集合中(算法第5行),否则需要继续遍历,找到包含子节点的子树,进行递归切分(算法第11行),最终得到子树节点列表,即子树序列$SubTrees$。

\begin{algorithm}[ht]  
	\renewcommand{\algorithmicrequire}{\textbf{Input:}}
	\renewcommand{\algorithmicensure}{\textbf{Output:}}
	\caption{Subtree partitioning algorithm $\left(split\_AST\right)$}  
	\label{alg2}
	\begin{algorithmic}[1]
    \Require Root node of abstract syntax tree:$Root\_Node$
    \Require Node set:$S$
		\Ensure SubTree set:$SubTrees$
    \State $SubTrees = \left\{\right\} $    
    \State SubTrees.append$\left(Root\_Node\right)$
		\For{child\_node $in$ Root\_Node.children}
      \If {child\_node $\in$ S}
        \State SubTrees.append$\left(child\_node\right)$
      \Else
        \For{subtree $in$ SubTrees}
          \If {child\_node $\in$ SubTrees.descendant}
            \State SubTrees.append$\left(child\_node\right)$
          \Else
            \State $ split\_AST\left(subtree\right)$
          \EndIf
        \EndFor
      \EndIf
    \EndFor \\
    \Return $SubTrees$
	\end{algorithmic}
\end{algorithm}

具体地,如图\ref{fig:astshili}所示,(a)为示例代码片段,(b)为利用代码分析工具Joern生成得到抽象语法树T,(c)为利用子树划分算法将其切分为一系列子树SubT。其中,以Method为根节点,其名称为main,body节点则代表了内部代码块,该块中包含三棵变量声明子树(Local)、一棵FOR循环条件子树(ForStatement)、一棵函数声明子树(Function)以及返回语句子树(ReturnStatement),而FOR循环条件子树内部包含一课If条件句法子树(IfStatement)。

\begin{figure}[H]
  \centering
  \includegraphics[width=0.95\textwidth]{figures/astshili.png}
  \caption{子树划分算法示例}\label{fig:astshili}
\end{figure}

% \lstset{language=C}
% \begin{lstlisting}
%     /* Count the sum of elements in an array */
%     #include <stdio.h>
    
%     int main()
%     {
%         int i = 0;//parameter one
%         int data[5]={1,2,3,4,5};//parameter two
%         int result = 0;//parameter three
%         for(;i<5;i++)
%         {
%             if(result < 10){
%                 result += data[i];
%             }
%         }
%         printf("%d",result); 
%         return 0;
%     }
% \end{lstlisting}

\subsection{抽象语法树表征学习设计}
\label{subsec:ASTModel}
(1)结构设计

为了提高抽象语法树维度代码表征能力,本文选取树卷积网络对上述得到子树序列进行建模,使用双向门控循环单元(BiGRU)对整个树进行建模。具体的模型设计如图\ref{fig:astmodel}所示。该模型主要包括输入层、子树卷积层、双向门控循环层(BiGRU)、最大池化层、输出层。
\begin{figure}[H]
  \centering
  \includegraphics[width=0.85\textwidth]{figures/astmodel.png}
  \caption{抽象语法树表征模型设计}\label{fig:astmodel}
\end{figure}

\ding{172}输入层:输入层用于向模型输入训练数据,在本方法中模型的输入为经过子树划分得到的子树序列。\ding{173}子树卷积层:针对子树序列中的每一个子树,通过基于树的卷积,捕获语句粒度语义信息。\ding{174}双向门控循环层:由两层GRU构成,同时捕获序列的双向语义信息。\ding{175}最大池化层:总结子树序列的输入特征,并将其缩减为一个单一的密集向量。\ding{176}输出层:每个抽象语法树对应一个输出。

(2) 模型选型

树表征阶段包含两个步骤:子树表征、整树表征。因此本文设计了两个神经网络模型:前者的输入是子树序列,针对每个子树进行基于树的卷积,在子树粒度上对代码进行信息提取;后者的输入是子树向量序列,针对所有子树的特征进行双向特征提取,并整合整棵树的信息,得到一个结构特征向量。

(2.1)子树表征模型

卷积神经网络(Convolutional Neural Network,CNN)是一种深度学习模型,其架构包括多个卷积层、池化层和全连接层。其中,卷积层负责从输入数据中提取特征,池化层用于降低数据维度,全连接层则用于将前面各层的特征映射到输出空间。其中,卷积思想是指将一个固定大小的窗口(通常被称为卷积核)在输入数据上按照一定的步长进行滑动,并对每个窗口中的局部片段进行特征提取,最后得到一系列特征,每个特征对应一个卷积核提取的特征。它能够将输入数据从底层到高层逐步抽象化,形成层次化的特征表示。

常见的卷积核通常都是正方形的,但树形结构通常是不规则,因此有研究\cite{8813290}提出了基于树的卷积神经网络。该网络的核心思想是将卷积操作扩展到树形结构上。通过定义在树上的卷积核,可以捕获树的节点与其邻居之间的局部特征。这种局部感知的方式与传统的卷积神经网络类似,但不同的是,其卷积核是三角形的,能够在不规则的树形结构上进行。基于树的卷积如下图\ref{fig:TreeBaseConvolution}所示。

\begin{figure}[H]
  \centering
  \includegraphics[width=0.65\textwidth]{figures/TreeBaseConvolution.png}
  \caption{子树表征:树卷积设计}\label{fig:TreeBaseConvolution}
\end{figure}

在基于树的卷积神经网络中,每个节点都与其邻居节点相连,形成一个局部邻域。卷积核在这个局部邻域上进行滑动,计算节点与其邻居的加权和,从而提取出特征。用数学公式表示,卷积核的滑动尺寸是$d$,它表示每个滑动窗口能够包括的树的层数。在这个窗口内包含$n$个节点$\left(x_1,x_2,\ldots,x_n\right)$,其中,$x_i \in \mathbb{R}^{N_{f}}$,$N_{f}$表示每个节点建模后的向量大小。使用公式\ref{e4.1}可以对滑动窗口内$n$个节点进行卷积得到基于树的卷积神经网络输出。
\begin{equation}\label{e4.1}
  \begin{split}
    y = \tanh \left(\sum_{i=1}^{n} W_{\text{conv}, i} \cdot x_{i}+b_{\text{conv}}\right)
  \end{split}
\end{equation}

其中,$y, b_{\text{conv}} \in \mathbb{R}^{N_{f}}, W_{\text{conv}, i} \in \mathbb{R}^{N_{c} \times N_{f}}$,$N_{c}$为最后卷积得到的向量长度。具体地,在上图\ref{fig:TreeBaseConvolution}的例子中,红色三角形表示卷积核,固定深度$d$为2,最后输出的卷积向量长度$N_{c}$为4。树结构在卷积前后保持相同的形状,而每个节点向量的维数由原来的3维变为4维。

基于树的卷积网络模型存在一个限制:其子节点的数量被限制为两个。但实际上,AST理论上可以存在无限数量的子节点,每次滑动窗口中的节点数$n$难以固定,从而导致无法确定权重矩阵的数量,即公式\ref{e4.9}中的$W_{conv,i}$。现有研究的解决方法是预先定义一组规则将抽象语法树转化为二叉树,然后再进行树卷积。这种处理方式,会改变源代码原有的语义,从而难以捕捉远程依赖关系,甚至丢失一些上下文信息,因此本文采用研究\cite{8813290}中提出的连续二叉树概念,将AST的每一个子树都看作二叉树,而不管它的形状和尺寸。图\ref{fig:Continuous}展示了连续二叉树的定义。

\begin{figure}[H]
  \centering
  \includegraphics[width=0.6\textwidth]{figures/Continuous Binary Tree.png}
  \caption{连续二叉树}\label{fig:Continuous}
\end{figure}

具体地,针对滑动窗口中的节点,需要设置三个变量:$W_{\text{conv}}^{t},W_{\text{conv}}^{l},W_{\text{conv}}^{r}$分别代表当前节点距离树形上、左、右三个方向,其卷积矩阵$W_{conv,i}$是这三个变量的线性组合,其系数根据节点在当前滑动窗口中的相对位置计算,具体公式为\ref{e4.2}:
\begin{equation}\label{e4.2}
  \begin{split}
    W_{\text{conv}, i}=\frac{d_{i}^{b}}{d_{i}^{b}+d_{i}^{t}} W_{\text{conv}}^{t}+\frac{d_{i}^{t}}{d_{i}^{b}+d_{i}^{t}} W_{\text{conv}, i}^{b}
  \end{split}
\end{equation}

其中,$d_{i}^{t}$表示当前节点距离树形顶点的距离,$d_{i}^{b}$表示当前节点距离树形左右顶点构成的边的距离,两者相加为$d$,即$d_{i}^{t} + d_{i}^{b} = d$。上式中$W_{\text{conv}, i}^{b}$有如下定义:

\begin{equation}\label{e4.3}
  \begin{split}
    \begin{array}{l}
      W_{\text{conv}, i}^{b}= 
      \left\{\begin{array}{ll}
      \frac{n_{i}^{r}}{n_{i}^{r}+n_{i}^{l}} W_{\text {conv}}^{l}+\frac{n_{i}^{l}}{n_{i}^{r}+n_{i}^{l}} W_{\text{conv}}^{r} & n_{i}^{l} \geq 1 \text { or } n_{i}^{r} \geq 1, \\
      \frac{1}{2} W_{\text{con}}^{l}+\frac{1}{2} W_{\text{conv}}^{r} & n_{i}^{l}=n_{i}^{r}=0 .
      \end{array}\right.
      \end{array}
  \end{split}
\end{equation}

其中,$n$表示树的同一层左右两个方向的兄弟节点个数。通过上述公式,可以计算在一个三角形的卷积窗口中,当节点越接近顶点,其$W_{\text{conv}}^{t}$的权重值越大,当节点越接近树形结构左下节点,其$W_{\text{conv}}^{l}$的权重值越大,当节点越接近树形节点结构右下节点,其$W_{\text{conv}}^{r}$的权重值越大。在具体实验中,本文设置滑动窗口的$d$值为2。

(2.2)整树表征模型

\ref{subsec:TokenModel}小节中提到长短期记忆网络LSTM模型针对长依赖问题效果明显,解决了传统RNN模型中存在的梯度消失问题,但是LSTM模型同样存在缺陷,其LSTM单元内包含三个门,机制相对复杂,计算成本会有所提高,因此,有研究提出了门控循环单元GRU,将遗忘门和输入门进行组装,形成一个新的更新门,这种改进使得GRU在模型参数和计算效率上通常优于LSTM。GRU单元的时序结构如下如\ref{fig:GRU}所示,每个GRU单元都包含两个门:重置门(Reset Gate)和更新门(Update Gate),通过重置门和更新门来直接控制信息的流动和隐藏状态的信息更新。

\begin{figure}[H]
  \centering
  \includegraphics[width=0.85\textwidth]{figures/GRU.png}
  \caption{GRU模型时序结构图}\label{fig:GRU}
\end{figure}

具体来说,首先会根据当前时间步的输入$x_{t}$和上一时间步的隐藏状态$h_{t-1}$计算出重置门和更新门的值,具体的计算公式见\ref{e4.4}。
\begin{equation}\label{e4.4}
  \begin{split}
    z_{t} = \sigma \left(W_{u} \cdot \left[h_{t-1},x_{t}\right] + b_z \right)
    \\
    r_{t} = \sigma \left(W_{r} \cdot \left[h_{t-1},x_{t}\right]  + b_r \right)
  \end{split}
\end{equation}

然后,输入重置门,对上一个时间步的隐藏状态进行处理,得到一个新的候选隐藏状态$\widetilde{h_t}$。其中,重置门用来控制上一个时间步的信息有多少应该被遗忘。具体地,通过公式\ref{e4.5}计算一个介于-1到1之间的值,决定上一个时间步的隐藏状态有多少信息应该被保留或遗忘,得出的值越接近0越有可能被丢弃,越接近1越有可能被记住。
\begin{equation}\label{e4.5}
  \begin{split}
    \widetilde{h_t} &= \tanh \left(W_x \cdot x_t \otimes U_h \cdot\left(r_t \cdot h_{t-1}\right) + b_h  \right)
  \end{split}
\end{equation}

接着,输入更新门,将新的候选隐藏状态$\widetilde{h_t}$和上一个时间步的隐藏状态$h_{t-1}$进行加权组合,得到当前时刻的隐藏状态$h_t$,如公式\ref{e4.6}所示。其中,更新门用来控制新输入的信息和上一个时间步的信息应该如何结合。具体地,通过公式\ref{e4.7} 决定当前时刻的隐藏状态应该有多少信息来自上一时刻的隐藏状态,以及有多少信息来自当前时刻的输入和重置门处理后的结果。通过更新门,GRU能够平衡新旧信息的影响,从而有效地捕捉序列数据中的长期依赖关系。最后,当前时刻的隐藏状态会被输出。

\begin{equation}\label{e4.6}
  \begin{split}
   h_t &= \left(1- z_t\right) \otimes h_{t-1} +  z_t \otimes \widetilde{h_t}
  \end{split}
\end{equation}

同样地,在抽象语法树中,某一个节点的状态不仅和前面层的节点状态有关,也可能和后一层的节点状态有关。为了更好地提高模型对AST子序列的表征能力,本文选择了双向BiGRU模型来进行树代码表征,模型的结构如下图\ref{fig:BiGRU}所示。BiGRU模型结构结合了双向RNN(双向循环神经网络)和GRU(门控循环单元)的优点,通过其双向结构能够更有效地学习并保持长期的信息流,从而使得当前时刻的输出与前后时刻的状态都产生联系,在处理长序列数据时表现出色,在具有较快收敛速度、较强学习能力等优点的同时,还能减少梯度消失问题。

\begin{figure}[H]
  \centering
  \includegraphics[width=0.5\textwidth]{figures/BiGRU.png}
  \caption{树表征:BiGRU模型设计}\label{fig:BiGRU}
\end{figure}

和\ref{subsec:TokenModel}提出的双向长短时记忆模型相比,GRU结构简单,参数简单,因此通常可以更快收敛到最优解,从而节省计算资源和时间。整个模型,除了子单元有所改变外,架构不变。这里不展开介绍,仅给出公式\ref{e4.7}。
\begin{equation}\label{e4.7}
  \begin{split}
    \overrightarrow{y_t} &= W_0 \cdot GRU\left(x_{t},\overrightarrow{h_{t-1}}\right) + b_0
    \\
    \overleftarrow{y_t} &= W_0 \cdot GRU\left(x_{t},\overleftarrow{h_{t-1}}\right) + b_0
    \\
    y_t &= \overrightarrow{y_t} \oplus\overleftarrow{y_t}
  \end{split}
\end{equation}


\section{AST表征方法具体实现}
\label{sec:ASTachieve}
在介绍具体实现之前,本节首先给出AST表征方法的输入:经过\ref{subsec:Preprocess}小节的代码预处理阶段,得到示例代码片段\ref{fig:code}对应的抽象语法树,如图\ref{fig:astcode}所示。仔细分析可以看出代码片段\ref{fig:ast1}对应的抽象语法树在FOR循环内部一共有13个子节点,子树高度为4;代码片段\ref{fig:ast2}对应的抽象语法树在WHILE循环内部也包含13个子节点,子树的高读为4,两者子节点个数相同;而代码片段\ref{fig:ast3}对应的抽象语法树在FOR循环中共有18个子节点,子树高度为6,与代码片段\ref{fig:ast1}对应的抽象语法树在根节点的下一层、下两层中1-9号节点架构相似。
\begin{figure}[htbp]
  \centering  %居中
  \subfigure[代码片段1对应的AST]{   %第一张子图
      \centering    %子图居中
      \includegraphics[width=0.3\textwidth]{figures/ast1}  
      \label{fig:ast1} %引用标签
  }
  \subfigure[代码片段2对应的AST]{ %第二张子图
      \centering    %子图居中
      \includegraphics[width=0.3\textwidth]{figures/ast2}
      \label{fig:ast2} %引用标签
  }
  \subfigure[代码片段3对应的AST]{ %第三张子图
      \centering    %子图居中
      \includegraphics[width=0.25\textwidth]{figures/ast3}
      \label{fig:ast3} %引用标签
  }
  \caption{示例源代码对应的抽象语法树}    %大图名称
  \label{fig:astcode}    %图片引用标记
\end{figure}

接下来,本章提出的基于子树划分的抽象语法树表征学习方法的实现如图\ref{fig:ast}所示。该方法的输入是一对代码片段$C_{a},C_{b}$对应的抽象语法树,表示为$AST_{a},AST_{b}$,输出是$C_{a},C_{b}$对应的结构特征向量 $V_{a}^{AST},V_{b}^{AST}$,整体采用Siamese架构,两个子网络共享权值,从下到上,主要包括子树划分、子树表征、树表征三个阶段。

\begin{figure}[H]
  \centering
  \includegraphics[width=0.9\textwidth]{figures/ast}
  \caption{基于子树划分的抽象语法树表征学习方法实现}\label{fig:ast}
\end{figure}

具体来说,在子树划分阶段,对于代码片段$C_{a}$,经过代码分析工具Joern生成得到抽象语法树$AST_{a}$,通过子树划分算法,可以得到子树序列,用$\left(SubTree_{1}^{a},SubTree_{2}^{a},\ldots,\notag \right.\\\left.SubTree_{1}^{m}\right)$表示,$m$是子树序列的长度。其子树划分过程可以表示为公式\ref{e4.8}:

\begin{equation}\label{e4.8}
  \begin{split}
    \left(SubTree_{1}^{a},SubTree_{2}^{a},\ldots,SubTree_{m}^{a}\right) = split\_AST \left(AST_{a}\right)
  \end{split}
\end{equation}

经过子树划分后,代码片段$C_{a}$生成了子树序列$\left(SubTree_{1}^{a},SubTree_{2}^{a},\ldots,SubTree_{m}^{a}\right)$。使用同样的划分方法,可以得到代码片段$C_{b}$生成了子树序列$\left(SubTree_{1}^{b},SubTree_{2}^{b},\ldots,\notag \right.\\\left.SubTree_{n}^{b}\right)$,$n$是代码片段$C_b$对应的子树序列长度。

在树表征阶段,首先将代码片段的子树序列作为输入,使用基于树的卷积神经网络模型进行编码,得到每个子树对应的语句向量$\left( e_{1}^{a},e_{2}^{a},\ldots,e_{m}^{a}\right)$。然后,使用双向门控循环单元(BiGRU)来模拟语句的自然性,通过最大池化层将BiGRU的隐藏状态采样到单个固定长度的向量$V_{a}^{AST}$中,作为最终的抽象语法树表示,即结构特征向量。具体的处理过程如公式\ref{e4.9}
\begin{equation}\label{e4.9}
  \begin{split}
    h_{1}^{aBiGRU},h_{2}^{aBiGRU},\ldots,h_{n}^{aBiGRU} = BiGRU \left(e_{1}^{a},e_{2}^{a},\ldots,e_{m}^{a}\right) \\
    V_{a}^{AST} = MaxPool \left( h_{1}^{aBiGRU},h_{2}^{aBiGRU},\ldots,h_{n}^{aBiGRU} \right)
  \end{split}
\end{equation}

同样,可以使用相同的计算以子树序列$\left(SubTree_{1}^{b},SubTree_{2}^{b},\ldots,SubTree_{n}^{b}\right)$作为输入为代码片段$C_{b}$计算$V_{b}^{AST}$。


\section{实验验证}
\label{sec:ASTExperiment}
为了验证基于子树划分的抽象语法树表征学习方法的有效性,本节开展实验验证。首先,介绍了实验的具体设计,接着对子树划分、树表征模型进行消融实验。

\subsection{实验设计}
\label{subsec:ASTDesign}

本节使用与\ref{sec:TokenExperiment}节中同样的实验环境和数据集对基于子树划分的抽象语法树表征学习方法进行对比实验。使用代码分析工具Joern获取数据集中代码片段的抽象语法树,然后使用基于树的卷积神经网络训练子树嵌入,并将嵌入向量大小设置为128。模型的batch\_size为32,优化器为Adam优化器,学习率为0.002,克隆检测的阈值设置为0.5。同样选取常用的精确率(Precision)、召回率(Recall)、F1值作为评估指标。

\subsection{实验结果}
\label{subsec:TokenResult}
消融对比实验:体现AST子树划分的有效性

基于AST的Tree-LSTM

基于AST的+子树划分的Tree-LSTM

\begin{table}
  \centering
  \caption{抽象语法树子树划分实验结果} %{tab:category}
  \begin{tabular*}{0.9\textwidth}{@{\extracolsep{\fill}}cccc}
  \toprule
    对比			&P		&R		&F1 \\
  \midrule
    基于AST的Tree-LSTM			&89.78	&87.88		&88.82 \\
    基于AST的+子树划分的Tree-LSTM			&92.7	&87.6		&90.0 \\
  \bottomrule
  \end{tabular*}
\end{table}

\section{本章小结}
\label{sec:Summary4}
本章主要对RLCCD中基于子树划分的抽象语法树表征学习方法的设计与实现进行详细阐述。首先介绍了抽象语法树维度的研究动机,其次介绍了抽象语法树表征学习的方法设计,具体论述了其整体框架、子树划分、树表征学习,接着开展实验验证,结果表明了此方法的有效性和模型的准确性。




\chapter{基于图过滤的程序依赖图表征学习}
\label{chap:PDG}
本章主要对本文提出的基于图过滤的程序依赖图表征学习方法进行详细介绍,首先介绍其基本思想,其次阐述其具体方法设计与实现,最后进行实验验证。

\section{研究动机}
\label{sec:PDGMotivation}

程序依赖图PDG是代码的一种图形表示,所含结构信息最多,能够表示程序的控制依赖,数据依赖等关系,是一种带有标记的有向多重图。程序依赖图PDG结点代表语句,边代表依赖关系,依赖关系包括数据依赖和控制依赖。基于图的代码表征方法首先使用代码分析工具构建包含代码语法结构、调用关系、数据流等信息的程序依赖图,然后通过子图匹配的方法,将PDG图中的控制流和数据流编码为一个紧凑的语义特征矩阵,其中每个元素都是一个高维的稀疏二值特征向量。通过将代码表示为图的形式使得模型能够更好地理解代码中不同部分之间的依赖关系,更适合研究代码内的丰富语义信息。

基于图的代码表征方法存在两种主要限制:

(1)规模开销较高:基于图的表征学习方法通常需要构建代码的结构图或控制流图作为分析的基础,对于具有复杂控制流或数据流的代码片段,构建准确的图表示是一个不小的挑战。特别地,当代码片段中具有循环、递归或异常处理机制时,图的构建过程更加困难。这些复杂结构不仅增加图构建的复杂性,还有可能导致图表示的精度下降;即使在成功生成图之后,图表征学习方法的计算成本也很高。例如,对图进行子图同构等操作时,往往需要借助复杂的图算法来实现,并将生成的程序依赖图两两匹配,对于包含$n$个代码片段的数据集,需要进行$n^2$次匹配检测。而其中包含大量无用的匹配,会浪费大量的时间和计算资源。这些算法不仅计算成本大,而且随着代码库规模的扩大,处理时间也会显著增加,导致开销高。

(2)对代码修改的敏感性:在实际开发过程中,代码通常会经过各种微小的更改,例如变量名更改、代码格式化、添加或删除注释等,这些修改可能会导致图的表示发生显著变化。具体来说,当代码中的变量名被更改时,图的节点和边可能会受到影响,因为变量名通常作为图中的一个重要特征被考虑在内。同样,代码格式的调整,如缩进、换行或空格的变化,虽然不影响代码的逻辑功能,但也可能导致图的拓扑结构发生变化。此外,添加或删除注释虽然对代码的执行没有影响,但在构建代码图时,这些注释也可能被当作图的一部分,从而影响到图的表示。因此,基于图的克隆检测方法可能无法准确地检测出这些轻微修改过的克隆代码。更进一步,随着代码库的不断增长和变化,基于图的克隆检测方法可能需要不断地更新和调整以适应新的代码结构。这意味着方法的实现和维护成本可能会相对较高,因为开发者需要定期更新和调整方法以适应代码的变化。这种持续的更新和调整不仅增加了工作负担,还可能影响到方法的长期有效性。

因此,针对上述问题,本文提出了一种基于图过滤的程序依赖图表征学习方法,该方法通过预处理图过滤,减少候选PDG对集合的规模。

\section{PDG表征方法方法设计}
\label{sec:PDG}
本节将介绍基于图过滤的程序依赖图表征学习方法设计与实现,首先介绍该方法的整体框架,并从图过滤、程旭依赖图表征学习两方面介绍具体设计。 

\subsection{框架概述}
\label{subsec:PDGOverview}
本文提出的基于图过滤的程序依赖图表征学习方法整体框架如图\ref{fig:pdgframework}所示。该框架的输入是代码片段对应的程序依赖图,输出是对应的语义特征向量,主要包括图过滤、图表征两个阶段。

\begin{figure}[H]
  \centering
  \includegraphics[width=0.95\textwidth]{figures/pdgframework.png}
  \caption{基于图过滤的程旭依赖图表征学习框架}\label{fig:pdgframework}
\end{figure}

首先,图过滤阶段以代码片段对应的程序依赖图作为训练数据,设计一个

其次,

在上述框架中,


\subsection{图过滤设计}
\label{subsec:PDGPreModel}

本文使用代码分析工具Joern生成程序依赖图。其中,Joern通过静态分析源代码,生成关键的图结构信息,反映代码中的依赖关系和函数调用层次,同时Joern提供查询和可视化功能,用户可以通过命令将分析结果导出为多种格式,从而更好地理解代码的逻辑和流程。使用Joern工具,生成代码片段\ref{fig:code1}对应的程序依赖图,并导出DOT文件如\ref{fig:pdgshili1}所示,使用开源图形可视化工具Graphviz对DOT文件进行转化,可以得到如\ref{fig:pdgshili2}的图形。

\begin{figure}[H]
  \centering
  \subfigure[代码片段1对应的PDG DOT格式]{   %第一张子图
      \centering    %子图居中
      \includegraphics[width=0.4\textwidth]{figures/pdgshili}  
      \label{fig:pdgshili1} %引用标签
  }
  \subfigure[代码片段1对应的PDG可视化]{ %第二张子图
      \centering    %子图居中
      \includegraphics[width=0.5\textwidth]{figures/pdgshili2}
      \label{fig:pdgshili2} %引用标签
  }
  \caption{程序依赖图DOT文件示例}
  \label{fig:pdgshili}
\end{figure}
% \lstset{language=C}
% \begin{lstlisting}
%   digraph "main" {  
%     "9" [label = <(METHOD,main)<SUB>4</SUB>> ]
%     "46" [label = <(METHOD_RETURN,int)<SUB>4</SUB>> ]
%     "12" [label = <(&lt;operator&gt;.assignment,i = 0)<SUB>6</SUB>> ]
%     "16" [label = <(&lt;operator&gt;.assignment,data[5]={1,2,3,4,5})<SUB>7</SUB>> ]
%     "25" [label = <(&lt;operator&gt;.assignment,result = 0)<SUB>8</SUB>> ]
%     "41" [label = <(printf,printf(&quot;%d&quot;,result))<SUB>14</SUB>> ]
%     "44" [label = <(RETURN,return 0;,return 0;)<SUB>15</SUB>> ]
%     "30" [label = <(&lt;operator&gt;.lessThan,i&lt;5)<SUB>10</SUB>> ]
%     "33" [label = <(&lt;operator&gt;.postIncrement,i++)<SUB>10</SUB>> ]
%     "45" [label = <(LITERAL,0,return 0;)<SUB>15</SUB>> ]
%     "18" [label = <(&lt;operator&gt;.arrayInitializer,{1,2,3,4,5})<SUB>7</SUB>> ]
%     "36" [label = <(&lt;operator&gt;.assignmentPlus,result += data[i])<SUB>12</SUB>> ]
%     "38" [label = <(&lt;operator&gt;.indirectIndexAccess,data[i])<SUB>12</SUB>> ]
%       "44" -> "46"  [ label = "DDG: &lt;RET&gt;"] 
%       "12" -> "46"  [ label = "DDG: i = 0"] 
%       "16" -> "46"  [ label = "DDG: data"] 
%       "16" -> "46"  [ label = "DDG: {1,2,3,4,5}"] 
%       "16" -> "46"  [ label = "DDG: data[5]={1,2,3,4,5}"] 
%       "25" -> "46"  [ label = "DDG: result = 0"] 
%       "30" -> "46"  [ label = "DDG: i"] 
%       "30" -> "46"  [ label = "DDG: i&lt;5"] 
%       ...
%       "30" -> "30"  [ label = "CDG: "] 
%       "30" -> "38"  [ label = "CDG: "] 
%       "30" -> "33"  [ label = "CDG: "] 
%       "30" -> "36"  [ label = "CDG: "] 
%     }
% \end{lstlisting}

%\notag \right.\\\left.

分析上图\ref{fig:pdgshili1},程序依赖图DOT文件的描述包含两个部分:使用\textquotedbl number \textquotedbl [label =  <\(function,name\)> ]来描述PDG点的特征,包括节点编号、节点的标签,其中标签内还包括源代码语句中的变量名称、变量属性等信息。使用\textquotedbl number \textquotedbl $\to$ \textquotedbl number \textquotedbl [label = \textquotedbl CDG/DDG:data \textquotedbl ]来描述PDG边的特征,包括边的起始点编号、边的终点标号、边代表的依赖关系(控制依赖用CDG表示、数据依赖用DDG表示)。使用开源图形可视化工具得到的图\ref{fig:pdgshili2}中包含13个顶点,用实线箭头表示数据依赖边,虚线箭头表示控制依赖边。

如果代码片段功能复杂,那么对应的程序依赖图规模也会很大,同时包含很多冗余边。例如上图\ref{fig:pdgshili2}中红框中的边,它们起始节点、终止节点均相同。右上角的红框对应\ref{fig:pdgshili1}中的data数组,数组内包含5个元素,对应含有5条边。实际上,这些边只有数值不同,属性相同,因此需要进行适当的优化来使得程序程序依赖图的结构精简,又不会丢失语义信息。

针对上述问题,本文设计了一种基于候选图对集合的图过滤算法,算法的伪代码如\ref{alg3}所示。该算法有多个输入:候选程序依赖图集合$G$、PDG有效行数阈值$L$、程序依赖图对规模比率$T$,输出为:经过过滤后的PDG对集合$R$,初该算法分为四个步骤:PDG图结构化简、规模过滤、非同构判断、数值特征过滤,每个步骤的作用如下:

\begin{algorithm}[ht]  
	\renewcommand{\algorithmicrequire}{\textbf{Input:}}
	\renewcommand{\algorithmicensure}{\textbf{Output:}}
	\caption{Graph filter algorithm $\left(filter\_PDG\right)$}  
	\label{alg3}
	\begin{algorithmic}[1]
    \Require PDG pairs:$G$
    \Require The threshold:$L$
    \Require The threshold of PDG pair's scale ratio:$T$
    \Require The threshold of CV's string numberical similarity:$G_s$
		\Ensure Candidate PDG pairs:$R$
    \State initialization
		\For{each PDG paris $G_1,G_2$  $in$ $G$}
      \State deleteSelfLoops($G_1$)
      \State deleteSelfLoops($G_2$) \Comment{step1:PDG图结构化简}
      \If {sizeof($G_1$) < L or sizeof($G_2$) < L}
        \State PDG pair($G_1,G_2$) is filtered \Comment{step2:规模过滤}
      \Else
        \If {min($G_1,G_2$) / max($G_1,G_2$) < T}
          \If{there is subgraph between ($G_1,G_2$ )} \Comment{step3:非同构判断}
            \State R $\leftarrow$ R $\cup \left(G_1,G_2\right)$ 
          \Else
            \State PDG pair($G_1,G_2$ ) is filtered
          \EndIf
        \Else
          \If {number similarity of $G_1,G_2$ > $G_s$} \Comment{step4:数值特征过滤}
            \State R $\leftarrow$ R $\cup \left(G_1,G_2\right)$
          \Else
            \State PDG pair($G_1,G_2$ ) is filtered
          \EndIf
        \EndIf
      \EndIf 
    \EndFor \\
    \Return $R$
	\end{algorithmic}
\end{algorithm}

(1)PDG图结构化简:

(2)规模过滤:

(3)非同构判断:

(4)数字特征过滤::


通过收集PDG的简单特征来过滤掉明显不可能为克隆的PDG对。具体的,根据PDG的节点个数、控制边数、执行边数、数据边数、声明节点数、函数调用数、传入参数、传出参数等代表特征进行过滤,在大幅减少候选PDG对规模的同时,保证真正的克隆对不会被过滤掉而导致整体克隆检出率的降低。


\subsection{程序依赖图表征学习设计}
\label{subsec:PDGModel}

(1)结构设计

为了提高程序依赖图维度代码表征能力,本文选取图卷积网络对上述得到程序依赖图进行建模。具体的模型设计如图\ref{fig:pdgmodel}所示。该模型主要包括输入层、多层卷积层、自注意力层、输出层。
\begin{figure}[H]
  \centering
  \includegraphics[width=0.85\textwidth]{figures/pdgmodel.png}
  \caption{程序依赖图表征模型设计}\label{fig:pdgmodel}
\end{figure}

(2)模型选型


对于提取到的程序依赖图,本课题拟通过图卷积神经网络将其转换成向量。


图卷积神经网络是一种特殊的前馈神经网络结构,为减少网络中参数个数,用卷积层来代替传统的全连接层,提高神经网络的训练效率,卷积神经网络可以提取信息最多的数据特征,生成一个固定大小的向量表示结构,从而挖掘深层次的语法和语义信息,在代码克隆检测任务中有较好的性能表现。

图卷积神经网络(GCN)的基本原理是通过卷积操作来提取图中节点的特征信息。在GCN中,节点的特征表示会考虑其邻居节点的特征,并通过学习得到的参数来更新。通过多层GCN的堆叠,可以逐步传播全局信息,实现对整个图的信息聚合和表示学习。GCN特别适合于那些节点表示和节点间关系都很重要的问题,如社交网络分析、分子结构识别、推荐系统等。


图中的每个结点无时无刻不因为邻居和更远的点的影响而在改变着自己的状态直到最终的平衡,关系越亲近的邻居影响越大。

GCN的主要思想:对于每个结点,我们都要考虑其所有邻居以及其自身所包含的特征信息。假设我们使用average()函数,那对每一个结点进行上述操作后,就可以得到能够输入到神经网络的平均值表示。




\section{PDG表征方法具体实现}
\label{sec:PDGachieve}
在介绍具体实现之前,本节首先给出PDG表征方法的输入:经过\ref{subsec:Preprocess}小节的代码预处理阶段,得到示例代码片段\ref{fig:code}对应的程序依赖图,如图\ref{fig:pdgcode}所示。图中的实线表示节点之间的控制依赖,虚线表示节点之间的数据依赖。仔细分析三张图,其中图\ref{fig:pdg1}和图\ref{fig:pdg2}红框中节点11、12虽然位置有所改变,但是其边依赖均相同,因此可以视为完全相同的同构图;而图\ref{fig:pdg3}因为增加了一个if语句,图中也增加了一个节点,同时添加的红色的线表示新增的数据依赖、控制依赖。

\begin{figure}[htbp]
  \centering  %居中
  \subfigure[C语言代码片段1对应的PDG]{   %第一张子图
      \centering    %子图居中
      \includegraphics[width=0.3\textwidth]{figures/pdg1}  
      \label{fig:pdg1} %引用标签
  }
  \subfigure[C语言代码片段2对应的PDG]{ %第二张子图
      \centering    %子图居中
      \includegraphics[width=0.3\textwidth]{figures/pdg2}
      \label{fig:pdg2} %引用标签
  }\subfigure[C语言代码片段3对应的PDG]{ %第二张子图
  \centering    %子图居中
  \includegraphics[width=0.3\textwidth]{figures/pdg3}
  \label{fig:pdg3} %引用标签
}
  \caption{示例源代码对应的程序依赖图}    %大图名称
  \label{fig:pdgcode}    %图片引用标记
\end{figure}

接下来,本章提出的基于图过滤的程序依赖图表征学习方法的实现如图\ref{fig:pdg}所示。该方法的输入是一对代码片段$C_{a},C_{b}$对应的程序依赖图,表示为$PDG_{a},PDG_{b}$,输出是$C_{a},C_{b}$对应的语义特征向量 $V_{a}^{PDG},V_{b}^{PDG}$,整体采用Siamese架构,两个子网络共享权值,从下到上,主要包括图过滤判断、图表征三个阶段。

\begin{figure}[H]
  \centering
  \includegraphics[width=0.9\textwidth]{figures/pdg.png}
  \caption{基于图过滤的程序依赖图表征学习方法实现}\label{fig:pdg}
\end{figure}

在图过滤阶段,

\section{实验验证}
\label{sec:PDGExperiment}
为了验证基于图过滤的程序依赖图表征学习方法的有效性,本文
\subsection{实验设计}
\label{sec:PDGDesign}
和3.3.1 相同
\subsection{实验结果}
\label{subsec:PDGResult}
消融对比实验:体现图过滤机制的有效性

基于PDG的GCN

基于PDG的+图过滤的GCN


\begin{table}
  \centering
  \caption{图过滤机制实验结果} %{tab:category}
  \begin{tabular*}{0.9\textwidth}{@{\extracolsep{\fill}}cccc}
  \toprule
    对比			&P		&R		&F1 \\
  \midrule
    基于PDG的GCN			&0.xx	&0.xx		&0.xx \\
    基于PDG的+图过滤的GCN			&0.xx		&0.xx		&0.xx \\
  \bottomrule
  \end{tabular*}
\end{table}

\section{本章小结}
\label{sec:Summary5}
本章主要对RLCCD中基于图过滤的程序依赖图表征学习方法的设计与实现进行详细阐述。首先介绍了程序依赖图维度的研究动机,其次介绍了程序依赖图表征学习的方法设计,具体论述了其整体框架、图过滤、图表征学习,接着开展实验验证,结果表明了此方法的有效性和模型的准确性。




\chapter{特征融合及RLCCD框架验证}
\label{chap:fusion}
本章主要对RLCCD方法中最后一环:基于多模态学习的特征融合方法进行介绍,同时针对RLCCD进行实验评估及验证。具体地,RLCCD是由上述基于预训练辅助模型的Token表征学习、基于子树划分的抽象语法树表征学习、基于图过滤的程序依赖图表征学习方法三种维度融合形成并进行实现的表征方法,本章首先详细介绍特征融合的两种方法,通过与SourcererCC、ASTNN、SCDetector进行对比实验以验证该框架的有效性。
\section{特征融合}
特征融合方法是指将不同来源或不同层次的特征进行组合,合并成一个比输入特征更具有判别能力的特征,该多维特征能够在低维空间中高效计算实体和关系的语义联系,提高特征的表达能力和分类效果,有利于下游代码克隆检测任务的学习。

因此,经过第\ref{chap:Token}、\ref{chap:AST}、\ref{chap:PDG}章的表征学习,得到三种维度的特征向量:属性特性$V^{Token}$、结构特征$V^{AST}$、语义特征$V^{PDG}$。上述三种表征方式得到的特征向量通常具有信息互补性,且不同维度的特征是代码表示的平行语料,具有信息等价性\cite{Multimodal},因此,本文提出了基于多模态学习的特征融合方法,将三个维度的特征向量通过特征融合生成多维表示。然后将多维表示送入单层线性网络。本文设计了两个特征融合方式:特征连接(Feature concatenation)和特征加法(Feature addition),如下图\ref{fig:concat_add}所示。

\begin{figure}[H]
  \centering
  \includegraphics[width=0.55\textwidth]{figures/concat_add.png}
  \caption{特征融合方法}\label{fig:concat_add}
\end{figure}

其中特征连接Concat是指直接将两个特征张量在某个维度上连接在一起,生成一个更大的张量。例如两个输入特征x和y的维数若为p和q,输出特征z的维数为p+q。在连接时,采取串联策略,两个张量的维度必须保持一致。而特征加法Add是指将两个特征张量按照元素相加在一起,生成一个新的张量。采取并行策略,特征向量本身的维度并没有增加。上述两种融合方式可以用公式\ref{e6.1}表示:

\begin{equation}\label{e6.1}
  \begin{split}
    \tilde{V_{concat}} &= concat \left( V^{\text{Token}} , V^{\text{AST}} , V^{\text{PDG}}\right) \\
    \tilde{V_{add}} &= add \left( V^{\text{Token}} , V^{\text{AST}} , V^{\text{PDG}}\right) \\
    V &= W_{dt} \cdot \tilde{V} + b_{dt}
  \end{split}
\end{equation}

其中$V$表示最终的多维混合代码表示,$W_{dt}$在生成的多维混合表示中平衡属性特征、结构特征、语义特征的组成,而$b_{dt}$在训练模型时使模型偏向最终收敛。

\section{RLCCD框架验证}
本文的实验设计主要围绕以下5个方面的研究问题:

• RQ1:本文提出的预训练辅助模型策略是否优于基线方法?

• RQ2:本文提出的子树划分策略是否优于基线方法?

• RQ3:本文提出的图过滤策略能否优于基线方法?

• RQ4:本文提出的特征融合方法是否优于单个维度方法?

• RQ5:本文提出的多维源代码表征学习方法,与现有代码克隆检测工具相比表现如何?

在上述5个方向的研究问题中,RQ1是针对Token维度优化策略的评估,在\ref{sec:TokenExperiment}小节中已经给出;RQ2是针对抽象语法树维度优化策略的评估,在\ref{sec:ASTExperiment}小节中已经给出;RQ3是针对程序依赖图维度优化策略的评估,在\ref{sec:PDGExperiment}小节中已经给出;RQ4是针对基于多模态学习的特征融合方法进行评估;RQ5从整体工具有效性的角度开展评估。

\subsection{实验设置}

本章实验均在Ubuntu 16.04 LTS(64位)系统下进行,其系统硬件配置与\ref{subsec:Environment}所述相同。本实验选用JieBa分词获取代码的Token序列,使用代码分析工具Joern获取代码的抽象语法树、程序依赖图。代码克隆检测模型的开发过程中使用到的编程语言L:Python 3.6、深度学习框架Pytorch1.10、机器学习库Scilit-learn 0.19。同时,为了验证RLCCD的可行性与有效性,本章实验仍选用与前述\ref{subsec:Dataset}相同的数据集,并按照3:1:1的比例将其划分为训练集、测试集、验证集,所有实验在训练集上训练模型,并选择在验证集上产生最佳F1的参数来评估模型在测试集上的性能。

\subsection{对比工具}

为了更好地对比整个方法的实验效果,本文选取近几年来各个维度较为先进、经典开源的工具,并在相同数据集上进行了检测结果的对比,具体工具主要如表\ref{tab:tool}所示。

\begin{table}[htp]
  \centering
  \caption{代码克隆检测实验对比工具介绍} 
  \label{tab:tool}
  \renewcommand{\arraystretch}{1.1}
  \begin{tabular*}{0.8\textwidth}{@{\extracolsep{\fill}}cc}
  \toprule
    方法名称			&类型介绍		\\
  \midrule
  SourcererCC		 &基于Token的克隆检测方法 \\
  ASTNN			     &基于AST的克隆检测方法 \\
  CCSharp		     &基于PDG的克隆检测方法 \\
  SCDetector		 &基于Token、PDG的克隆检测方法 \\
  RLCCD(本文方法)	&基于Token、AST、PDG多种维度的克隆检测方法 \\
  \bottomrule
  \end{tabular*}
\end{table}

SourcererCC\cite{7886988}:SourcererCC是一种相对较新的基于token的克隆检测工具。该工具通过词袋模型,把收集的数据全部编码成词频信息,然后将代码行转换成一个由词频构成的向量,通过向量的比较获取相似度。

ASTNN\cite{8812062}:ASTNN是一种基于神经网络的源代码表示方法。它将整个抽象语法树AST分解成一系列小型语句子树,并通过捕获语句的词法和语法信息将语句子树分别编码为向量,最后采用了RNN模型生成代码片段的向量表示。ASTNN方法完整保留了抽象语法树的结构信息,能够检测到所有类型的代码克隆。

CCSharp\cite{9286111}:CCSharp是一种基于程序依赖图的代码克隆检测方法。它首先生成了源代码中函数级别的PDG,并对生成的图结构进行约简,随后进行特征向量的提取和过滤,最后应用Weisfeiler-Lehman图核算法进行图相似性比较找出代码克隆对。

SCDetector\cite{10.1145/3324884.3416562}:SCDetector一种是基于Token和图结合的方法。给定一个方法源代码,首先生成CFG,然后应用中心性分析将图转换为某些语义标记(即具有图细节的标记)。最后,这些语义标记被送到Siamese网络中,以训练模型并使用它来检测代码克隆对。

为了验证本文所提方法在克隆上的有效性,所有工具试实验结果均采用其参考文献中所提供的最好效果的参数配置,并统一在相同的实验数据集POJ104上进行对比实验。

\subsection{特征融合方法实验结果}

本节回答RQ4,本文提出的特征融合方法是否由于单个维度的表征方法?为了回答这个问题,本实验将Token、AST、PDG三个维度的最佳实验结果与多维融合方法Concat、Add进行对比。表\ref{tab:concat}展示了Token维度、树维度、图维度、采用Add方法融合三种维度、采用Concat方法融合三种维度共5种方法的准确率、召回率和F1得分。

\begin{table}[htp]
  \centering
  \caption{特征融合方法对实验结果的影响}
  \label{tab:concat}
  \begin{tabular*}{0.9\textwidth}{@{\extracolsep{\fill}}cccc}
  \toprule
    对比			& 准确率P(\%) & 召回率R(\%) & F1值(\%)  \\ 
  \midrule
    Token维度			   & 89.78	  & 95.62	  & 92.61	   \\  
    树维度		       & 92.75	  & 87.62	  & 90.11   \\ 
    图维度			     & 89.53		& 81.71		& 85.44   \\
    Add融合			     & 92.89		& 90.35		& 91.60   \\
    Concat融合     	 & 93.82		& 92.27		& 93.04  \\
  \bottomrule
  \end{tabular*}
\end{table}


基于对表\ref{tab:concat}数据的分析,可以得到以下结论:

(1)综合三种评估指标,基于多模态学习的特征融合方法的表现比单个维度的代码表征学习的表现优秀,其准确率、召回率、F1值均在92\%以上。说明基于多模态学习的特征融合方法可以利用数据之间的信息互补性,相较于单个维度,多维表征学习可以学习到更好的特征表示,从而有利于下游代码克隆检测任务的学习。

(2)Concat特征融合方法整体比Add融合方法表现优秀。深究其原因,Add特征融合相当于加入一种先验知识,对原始特征进行人为的特征融合。它描述的代码特征信息量增多,但维度本身并没有增加,当两个相加的特征向量不具备相同特征含义的时候,反而可能带来信息损失。而Concat特征融合是将三个张量拼接在一起,特征维度增加,每一特征下的信息并没有增加。这意味着Concat可以扩展特征的空间,让模型能够看到更多的信息和变化。通过将不同来源的特征拼接在一起,Concat可以促进特征之间的交互和整合,进一步提高模型的表现,使模型同时关注代码的局部和整体信息,因此具有更好的代码克隆检测能力。

(3)图维度的代码表征方法在代码克隆检测任务上的表现最差,三种指标均未超过90\%。深究其原因,可能是本文提出的基于图过滤的程序依赖图表征学习方法大大降低了图卷积神经网络模型的输入规模,数据量相比于基于Token的和基于树维度的表征方法少一个量级。当输入规模减少时,模型可能需要更长时间才能收敛到最优,同时模型可能无法充分学习数据的特征,影响下游任务表现。

\subsection{RLCCD性能评估实验结果}

本节回答RQ5,本文提出的RLCCD方法与现有代码克隆检测工具相比表现如何?为了回答这个问题,本实验RLCCD与Token维度方法SourcererCC、树维度方法ASTNN、图维度方法CCSharp、Token与图混合方法SCDetector进行对比,实验结果对比如表\ref{tab:RLCCD}所示。

\begin{table}[htp]
  \centering
  \caption{RLCCD性能评估实验结果}
  \label{tab:RLCCD}
  \begin{tabular*}{0.9\textwidth}{@{\extracolsep{\fill}}ccccc}
  \toprule
    维度 &对比工具		& 准确率P(\%) & 召回率R(\%) & F1值(\%)  \\ 
  \midrule
  \multirow{2}{*}{Token} & SourcererCC		& 11.23	  & 43.52		& 17.84 \\
  & RLCCD\_Token    & 89.78	  & 95.62	  & 92.61	   \\ 
  \cmidrule{2-5} 
  \multirow{2}{*}{AST}   & ASTNN	        & 87.92		& 95.47		& 91.54 \\
  & RLCCD\_AST      & 92.75	  & 87.62	  & 90.11   \\ 
  \cmidrule{2-5} 
  \multirow{2}{*}{PDG}   & CCSharp			  & 92.21	  & 52.13	  & 66.61 \\
  & RLCCD\_PDG		  & 89.53		& 81.71		& 85.44   \\
  \cmidrule{2-5} 
  \multirow{3}{*}{多维度} & SCDetector		 & 97.02	 & 81.05	 & 88.32 \\
  & RLCDD\_Add			& 92.89		& 90.35		& 91.60   \\
  & RLCDD\_Concat		& 93.82		& 92.27		& 93.04  \\
  \bottomrule
  \end{tabular*}
\end{table}


基于对表\ref{tab:RLCCD}数据的分析,可以得到以下结论:

(1)本文提出的RLCCD在代码克隆检测任务中表现最好,其准确率、召回率、F1值均在92\%以上。说明RLCCD方法可以利用Token、AST、PDG三个维度数据之间的信息互补性,学习到更好的特征表示,代码信息利用率高,从而有利于提高下游代码克隆检测任务的精度。

(2)SourcererCC在代码克隆检测任务中表现最差。深究其原因,SourcererCC采用了一种优化的倒排索引方法来快速查询代码块中潜在的克隆,这种方法相对简单,主要反映了源代码的词汇级信息,但缺乏足够的代码的结构信息、语义信息。同时,由于代码关注词汇级别的相似度,因此会误报一些仅仅词汇相似但功能完全不同的代码克隆对,从而降低了检测的精度。

(3)ASTNN方法的召回率最高,达到了95.47\%。深究其原因,ASTNN同样采用了抽象语法树子树划分的方法,将每个大的AST语法树划分为语句树,通过捕获语句的词汇和句法知识将语句树编码为向量,与本文提出的RLCCD方法不同,ASTNN直接使用了基于递归神经网络RvNN的语句编码器学习语句的向量表示,使其能够适应不同大小和形状的树状数据,不必局限于数据的固定树状结构,具有很好的灵活性。ASTNN通过递归的方式将网络应用于数据的不同层次,从而捕捉复杂的层次结构和依赖关系,面向下游代码克隆检测任务的召回率达到最高。

(4)CCSharp方法的召回率相较于准确率表现很差,只有52.13\%。深究其原因,CCSharp是一种基于Weisfeiler-Lehman图核算法的代码克隆检测方法,其迭代过程中过于关注图结构信息,反而忽略了节点自身的特征信息,同时对迭代次数敏感,次数过少无法充分捕获图的结构信息。

(5)SCDetector方法的准确率最高,达到了97.02\%。说明多个维度的代码特征融合相较于单个维度,正确识别代码克隆对的能力优秀。深究其原因,SCDetector融合了Token和图两种不同层次的代码表示方法,从而能够同时捕获代码在词汇级别和结构级别上的相似性。因此,面对代码片段存在词汇上相似,或者结构上相似但在词汇上有所变化的情况,SCDetector依旧可以有效识别。

\section{本章小结}
本章节主要对RLCCD框架中基于多模态学习的特征融合方法进行了介绍,然后进行了实验验证。首先介绍了两种特征融合方法:Add、Concat,然后介绍了实验设置和对比工具,通过单个维度与多维源代码表征学习方法RLCCD的对比实验证明了RLCCD的有效性,最后通过将RLCCD与另外4种代码克隆检测方法进行了对比实验,证明了RLCCD具有较好的性能表现。





\backmatter

% 结论
\begin{conclusion}

表征学习是指学习数据的表示,使其在构建分类器或其他预测因子时更容易提取有用信息。目前代码表征学习方法被用于代码克隆检测、代码搜索、代码补全等多个代码分析任务中,取得了一定的成就。然而,现有的代码表征学习仍然面临着信息利用不充分的问题,如何解决这一难题成为研究热点。

针对上述问题,本文提出了面向代码克隆检测的多维源代码表征学习方法RLCCD,实现对代码信息的充分利用,以更加全面准确与智能化的方式提高代码克隆测试效率。本文主要工作和成果如下:

(1)提出基于预训练辅助模型的Token表征学习

针对目前现有的基于Token的表征学习方法通常将词汇单元规范化,在后续神经网络模型训练过程中,当出现某个词汇在词汇表中没有出现过的难题,提出了一种基于预训练辅助模型的Token表征学习方法。该方法在模型训练之前,通过选取预训练辅助模型从代码语料库中学习基本单元的语法语义信息,
设计了一种相邻单元迭代算法,通过组合Token序列中相邻单元构造新的代码表示单元,不断更新词汇表,从而减少出现集外词问题的概率。在POJ104数据集上的消融实验评估表明,预训练辅助模型生成的词嵌入向量作为下游任务的输入特征,可以提升神经网络模型的检测能力,面向代码克隆检测任务,模型的F1值达到了92.61\%。

(2)提出基于子树划分的抽象语法树表征学习

针对现有的基于树的表征学习方法通常将抽象语法树转换为完整二叉树,可能破坏源代码原有语法结构,增加AST高度,丢失长期上下文信息,削弱了神经网络模型捕捉更真实和复杂语义能力,导致梯度消失的难题,提出了一种基于子树划分的抽象语法树表征学习方法。该方法将每个大型的AST分割成子树序列,并输入基于树的卷积神经网络中生成子树的嵌入,输入双向GRU模型中获得代码片段的结构向量表示。在POJ104数据集上的消融实验评估表明,子树划分方法在代码克隆检测的F1值提升了3.41\%。


(3)提出基于图过滤的程序依赖图表征学习

针对现有的基于图的表征学习方法通常将程序表征为有向多重图,继而采用图匹配算法将图中的控制流和数据流编码为一个紧凑的语义特征矩阵,矩阵中每个元素都是高维系数特征向量,所消耗的时间、空间开销巨大的难题,提出了一种基于图过滤的程序依赖图表征学习方法。该方法通过设计一个基于候选程序依赖图对的图过滤算法,减少候选图对的规模,提高后续模型的训练能力。具体的,通过PDG图结构化简、规模过滤、非同构判断、数值特征过滤四个步骤进行过滤,从而减少候选PDG对规模。在POJ104数据集上的消融实验评估表明,经过过滤后的PDG对占原来PDG对数的比例大致在15\%以下,能够有效减少时间、空间开销,提高代码克隆检测准确率。


(4)实现了面向代码克隆检测的多维源代码表征学习方法RLCCD

针对单个维度对代码信息利用不充分的问题,提出了基于多模态学习的特征融合方法,通过融合Token、AST、PDG三个维度的代码特征,采用Concat、Add两种不同的特征融合方法,从多维数据中学到更好的特征表示。同时,选取了代码克隆检测领域常见的基准集POJ104进行实验验证,并与现有开源的SourcererCC\cite{7886988}、ASTNN\cite{8812062}、CCSharp\cite{9286111}、SCDetector\cite{10.1145/3324884.3416562}方法进行比较,实验结果验证了RLCCD的可行性和有效性。

虽然本文所提出的面向代码克隆检测的多维源代码表征学习方法在现有的数据上表现良好,但是尚处于初步研究阶段,未来可以从以下几个方面展开进一步的研究。(1)本文提出的表征学习方法仅仅针对C语言进行了代码克隆检测任务实现,未来可以在此基础上针对不同的编程语言(Python、Java等)开展实验验证,进一步验证该方法的扩展性。(2)本文基于多模态表征学习的特征融合方法虽然在实验准确率上有所提升,但提升效果不明显。目前选取的特征融合方法只选取了Concat、Add两种,这不一定是最佳方法,未来工作中可以尝试其他特征融合方法,甚至针对不同维度的特征向量进行权重赋值,并尝试不同的网络模型结构,进一步提升该方法的效率。(3)目前现有的代码表征质量的评价标准没有统一,现有的工作都是基于具体的代码分析任务进行评价的\cite{张祥平_2011}。不同的分类任务中存在多种不同的指标,这种基于具体任务的结果来判断代码表征质量优劣的方式,并不是直接对表征向量的质量进行衡量。未来的工重点研究如何构建代码表征数据集,并在统一的评价标准上进行方法比较。

\end{conclusion}

% 参考文献
\input{./misc/2_reference.tex}

% 附录
% \input{./misc/3_appendices.tex}

% 个人成果
% 1. 在 `./reference/pub.bib` 中添加数据。
% 2. 在下方 `\addpubs` 添加该文献(参考下方示例)。

% **注意:如果发现渲染出来的文献编号不正确,请使用 `latexmk -c` 清除缓存后重新编译。**

\begin{publications}

  \addpubs{myCiteKey,myCiteKey2,dummy:1,dummy:2}

  % 主要针对硕士生
  \printbibliography[heading=none,category=mypub,resetnumbers=true]

  % 如果想要分为多个列表,可以使用以下的命令。
  % 主要针对博士生。
  % \pubsection{文章}
  %
  % \printbibliography[heading=none,type=article,category=mypub,resetnumbers=true]{}
  %
  % \pubsection{一些书}
  %
  % \printbibliography[heading=none,type=book,category=mypub,resetnumbers=true,notkeyword=dummy]{}
  %
  % \pubsection{另一些书}
  %
  % \printbibliography[heading=none,type=book,category=mypub,keyword=dummy,resetnumbers=true]{}

\end{publications}



% 致谢
\begin{acknowledgements}

  时间飞逝,转眼临近研究生毕业。当这篇文章正式落笔,就意味着我在北京理工大学的时光走到了尾声。回顾这段时间的工作学习, 在课堂上学到的理论知识 被强化、加深、理解,从而提高了我的专业知识素养,成为我人生的重要财富。在这里,我要向所有帮助过我的老师同学表示衷心的感谢。
  
  首先,非常真挚地感谢我的导师马锐。马老师,从毕业设选题、撰写开题报告、资料整理、进度反馈、论文修改的过程中,一直耐心指导,不厌  其烦地帮助我克服困难。马老师也腾出了很多个人时间,给我指导,使我可以顺利完成我的毕业论文。在这里,我真诚地祝愿马老师工作顺利、身体健康、万事顺意。

  感谢北京理工大学对我的培养教育,感谢它为我创造了一个环境优美、治学严谨、学风浓厚的校园氛围。在我迷茫时,可以去图书馆寻找知识解答疑惑;当我伤心时,可以去中心公园散步平复心情得到治愈;当我开心时,可以去食堂选一份美食来庆祝。我在北京理工大学度过了愉快、幸福的三年,在这里,也希望北理可以不断进步,未来我可以重回校园,为母校的发展感到无比自豪。

  感谢父母对我的支持和照顾,感谢他们多年来无微不至的照顾和坚定不移的支持。不管我在哪里,他们永远是我最坚强的后盾。当我遇到困难时,给我最温柔的建议和最坚定的鼓励。无论是在物质上,还是精神上,他们的支持与付出,鼓励着我坚定求学的决心,无畏地继续奔向美好地未来。
  
  感谢同学对我的鼓励陪伴 是大家让我收获到了友谊。我们在外出求学的生涯里相互扶持,用真挚的心换来坚定的情。虽然我们来自五湖四海,但我们在北京理工大学相遇,北京理工大学也见证了我们一起走过的四年。

  最后,再次向各位表示真挚的谢意,感谢在我前进的路上,有你们的参与
\end{acknowledgements}

% 个人简介(仅博士生需要此项)
% \begin{resume}

本人…。

\textcolor{blue}{
  硕士学位论文不必提供作者简介。博士学位论文应该提供作者简介,主要包括:姓名、性别、出生年月、民族、出生地;简要学历、工作经历(职务);攻读学位期间取得的其他研究成果或奖励。
}

\end{resume}


% 加入目录
\end{document}
